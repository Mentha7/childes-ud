% Appendix A
 
\chapter{Morphosyntactic Mappings}
\label{appendixa}

\section{Part-of-Speech Mapping}

\begin{table}[h!]
\caption {Complete Set of Universal POS Tags} \label{tab:uposset}
\begin{tabular}{@{}lllll@{}}
\toprule
\textbf{Universal POS Tags} & \textbf{Description}\\ \midrule
ADJ & adjective\\
ADV & adverb\\
INTJ & interjection\\
NOUN & noun\\
PROPN & proper noun\\
VERB & verb\\\midrule
ADP & adposition\\
AUX & auxiliary verb\\
CCONJ & coordinating conjunction\\
DET & determiner\\
NUM & numeral\\
PART & particle\\
PRON & pronoun\\
SCONJ & subordinating conjunction\\\midrule
PUNCT & punctuation\\
SYM & symbol\\
X & other\\\bottomrule
\end{tabular}\\
\vspace{0.5cm}\\
All UPOS tags except \texttt{SYM} are used in chatconllu's POS mappings.\\
\end{table}

\newpage
\begin{table}[htp!]
\caption {chatconllu Part-of-Speech Mapping 1} \label{tab:posmap1}
\begin{tabular}{@{}lllll@{}}
\toprule
\textbf{MOR POS Code} & \textbf{Description} & \textbf{UPOS}\\ \midrule
adj & adjective & ADJ\\
adj:pred & adjective-predicative & ADJ\\\midrule
adv & adverb & ADV\\
adv:tem & adverb-temporal & ADV\\
neg & negative & ADV\\\midrule
co & communicator & INTJ\\
fil & filler & INTJ\\
wplay & wordplay & INTJ\\
bab & babbling & INTJ\\
sing & singing & INTJ\\\midrule
qn & quantifier & DET\\
det:art & determiner-article & DET\\
det:dem & determiner-demonstrative & DET\\
det:int & determiner-interrogative & DET\\
det:num & determiner-numeral & DET\\
det:poss & determiner-possessive & DET\\
quant & quantifier & DET\\
art & article & DET\\
prepart & preposition with article & DET\\\midrule
aux & auxiliary & AUX\\
v:aux & verb-auxiliary & AUX\\
cop & verb-copula & AUX\\
mod & verb-modal & AUX\\\midrule
post & postmodifier & ADP\\
prep & preposition & ADP\\
vpfx & preverb/verbal particles & ADP\\\midrule
part & particle & PART\\
inf & infinitive & PART\\\midrule
coord & coordinator & CCONJ\\\midrule
conj & conjunction & SCONJ\\
comp & complementizer & SCONJ\\\midrule
num & numeral & NUM\\\bottomrule
\end{tabular}\\
\vspace{0.5cm}\\
All MOR part-of-speech categories handled by the current version of chatconllu (part 1).\\
\end{table}
\newpage
\vspace{-2cm}
\begin{table}[h!]
\caption {chatconllu Part-of-Speech Mapping 2} \label{tab:posmap2}
\begin{tabular}{@{}lllll@{}}
\toprule
\textbf{MOR POS Code} & \textbf{Description} & \textbf{UPOS}\\ \midrule
n & noun & NOUN\\
n:let & noun-letter & NOUN\\
n:pt & noun-plurale tantum & NOUN\\
on & onomatopoeia & NOUN\\
onoma & onomatopoeia & NOUN\\\midrule
pro:dem & pronoun-demonstrative & PRON\\
pro:exist & pronoun-existential & PRON\\
pro:indef & pronoun-indefinitive & PRON\\
pro:int & pronoun-interrogative & PRON\\
pro:obj & pronoun-object & PRON\\
pro:per & pronoun-personal & PRON\\
pro:poss & pronoun-possessive & PRON\\
pro:refl & pronoun-reflexive & PRON\\
pro:rel & pronoun-relative & PRON\\
pro:sub & pronoun-subject & PRON\\\midrule
n:prop & proper noun & PROPN\\\midrule
v & verb & VERB\\
vimp & verb imperative & VERB\\\midrule
end & MOR punctuation mark & PUNCT\\
beg & MOR punctuation mark & PUNCT\\
cm & MOR punctuation mark & PUNCT\\
bq & MOR punctuation mark & PUNCT\\
eq & MOR punctuation mark & PUNCT\\
bq2 & MOR punctuation mark & PUNCT\\
eq2 & MOR punctuation mark & PUNCT\\\midrule
none & none & X\\
dia & dialect & X\\
test & test words like "wug" & X\\
meta & metalinguistic use & X\\
phon & phonologically consistent form & X\\
fam & family-specific form & X\\
uni & Unibet transcription & X\\
L2 & second-language form & X\\
neo & neologism & X\\
chi & child-invented form & X\\\bottomrule
\end{tabular}\\
\vspace{0.5cm}
\\
All MOR part-of-speech categories handled by the current version of chatconllu (part 2).\\
\end{table}

