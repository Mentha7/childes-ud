\chapter{CHILDES \emph{CHAT} and UD \emph{CoNLL-U}} % Main chapter title

\label{Chapter2} % For referencing the chapter elsewhere, use \ref{Chapter2}

This chapter introduces the two annotation schemes in detail and gives the
reader an idea of what information is shared by both of the two annotations and
what information is exclusively relevant to only one of them. This will shed
light on the design and implementation of \emph{chatconllu} which will be
discussed in the following chapters.

\section{CHILDES \emph{CHAT} Transcription System}
\subsection{Motivation for \emph{CHAT} System}
The \emph{CHAT} (\textbf{C}odes for \textbf{H}uman \textbf{A}nalysis of
\textbf{T}ranscripts) transcription system \cite{Macwhinney2000} is a
standardised format devised for the TalkBank database. \emph{CHAT} transcripts
can be analysed by computer programs like \emph{CLAN} (\textbf{C}omputerized
\textbf{L}anguage \textbf{AN}alysis) to facilitate the transcription, sharing
and analysis of human interactions.

\subsection{File Structure}
For a \emph{CHAT} file to be well-formed, there are some basic structural
requirements to comply with. The minimum standards are prescribed by
\emph{minCHAT}.\\
% describe only relevant minCHAT structure
First of all, each \emph{CHAT} file should have 2 major parts: headers (that
stores metadata like language and participants) and utterances which are the
actual sentences being said. Each sentence is represented by a single line
beginning with an asterisk symbol \textbf{*}, this is called the main line.
Additional information can be optionally added. Annotations of morphological,
grammatical, phonological information or general remarks of the transcriber's
choice go into the dependent tiers of the main utterance.
% describe mor and gra tiers
\subsubsection{Dependent Tiers}
Almost any type of information can be organised into separate dependent tiers
and added to each utterance. The two most commonly seen tiers are \emph{\%mor}
and \emph{\%gra} which are automatically generated by the use of \emph{MOR},
\emph{POST}, \emph{POSTMORTEM} and \emph{MEGRASP}.\\
Since information in \emph{\%mor} and \emph{\%gra} should be kept with
individual tokens in \emph{CoNLL-U} files, they have to be processed such that
all relevant information from both tiers (if they are present) are associated
with the respective token in the utterance.
\subsubsection{Morphosyntactic Coding with \emph{\%mor} Tier}
% token links %mor to the main line. one-to-one correspondence between
normalised utterance and %mor.
% structure of individual words
% structure of word groups  -> conllu: multi-word tokens

\subsubsection{Syntactic Dependency Analysis with \emph{\%gra} Tier}

\section{UD \emph{CoNLL-U} Annotation Scheme}

\section{Shared Information vs. Exclusive Information}
% token as the basic unit of CHAT and CoNLL-U annotations.
% token also links %mor and %gra to the main line. one-to-one correspondence
between normalised utterance and %mor.
