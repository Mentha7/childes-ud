\chapter{CHILDES \emph{CHAT} and UD \emph{CoNLL-U}}
\label{Chapter2}

This chapter introduces the two annotation schemes of interest---\emph{CHAT} and \emph{CoNLL-U}---in detail and gives the reader an idea of what information is shared by both annotations and what information is exclusively relevant to only one of them. This will shed light on the implementation of \emph{chatconllu} which will be discussed in \Cref{Chapter3}.

\section{CHILDES CHAT Transcription System}

\subsection{Motivation}
The \emph{Codes for Human Analysis of Transcripts} (CHAT) transcription system is developed for the \emph{Child Language Data Exchange System} (CHILDES) project (\cite{Macwhinney2000}) to aid child language acquisition research.

As MacWhinney pointed out in the CHAT manual, collecting audio data for spontaneous conversational interactions is easy, one just have to turn on the recorder, however, processing the collected recordings and turning them into usable data for linguistic research is a demanding task (\cite{Macwhinney2000}).

For studies that rely on speech corpora, transcribing the conversations is an indispensable preprocessing step. Consistent treatment of raw data is critical to scientific research. Therefore, it is hard to imagine how difficult it was to maintain transcription standards when there was no standard format, especially for child speech, which is more chaotic than average adult speech data. Thus, CHAT is designed to accommodate the need for a standard transcription system of conversational speech. Moreover, CHAT transcripts can be analysed by TalkBank programs \sidenote{\url{https://talkbank.org/software/}} like \emph{Computerized Language Analysis} (commonly referred to as the CLAN Program), which facilitates the transcription, sharing and analysis of human conversational interactions.

\subsection{File Structure}
For a CHAT file to be well-formed, there are some basic structural requirements. The minimal structure of a well-formed CHAT file is described by \emph{minCHAT} (\cite{Macwhinney2000}, pp. 21-22).

% describe only relevant minCHAT structure
First of all, each CHAT file should have at least two parts:
\begin{itemize}
    \item \emph{headers} preceded by the \texttt{@} symbol that stores metadata like language and participants. Some headers are obligatory \sidenote{Obligatory headers include: the file-initial header @Begin, @Languages, @Participants, @ID and the file-final header @End.}, some are optional.\\
    Despite its name, headers can be added in the middle of a CHAT file to represent information like tape locations. It is likely that \emph{header} is a misnomer, and they should be called \emph{meta} as they gernally preserve higher-level information and are not confined to file-inital positions.
    \item the \emph{main line}---utterances led by the asterisk symbol \texttt{*}, which are the actual sentences being said.
\end{itemize}

Then, additional information can be \emph{optionally} added as dependent tiers. Annotations of morphological, syntactical, and phonological information or general remarks of the transcriber's choice all go into separate dependent tiers of the main utterance. Dependent tiers are preceded by the percent symbol \texttt{\%}.

% describe mor and gra tiers
\subsection{Dependent Tiers}
The inclusiveness and extensibility of the CHAT format is demonstrated by the use of dependent tiers, since almost any type of information can be organised into separate dependent tiers and added to the relevant utterances. The CHAT manual defines a standard list of dependent tiers (\cite{Macwhinney2000}, pp. 81-87). The tier names are mostly lowercase three-letter codes, but with the exceptions of tier names preceded by an \texttt{x}, like \texttt{\%xmor}. With the active use of dependent tiers, researchers can adjust the level of specificity in the CHAT files to meet their individual needs. At the same time, since dependent tiers are optional, they do not interfere with other obligatory parts of the file.

A dependent tier can contain one of two types of information:
\begin{itemize}
    \item utterance-level information
    \item word-level information
\end{itemize}

\paragraph{utterance-level information}
Utterance-level information usually appears as descriptive remarks. For example, the action tier \texttt{\%act} that describe actions by the speakers is shown in the example below \sidenote{Excerpt from Adam/040217.cha of the Brown Corpus (\cite{brown1973})}.\\

\lstset{
numbers = none,
frame = single,
}

\begin{lstlisting}[caption={Example of a dependent tier with utterance-level information.}, label={lst:chatsent1}]
*PAU:   0 .
%act:   hits Adam
\end{lstlisting}


\paragraph{word-level information}
Word-level information is associated with words in the utterance. Word-based strings encoded with information are separated by the space character. The below example \sidenote{from the same file in the Brown Corpus as Listing \ref{lst:chatsent1}} shows two dependent tiers with word-level information.

\clearpage

\lstset{
numbers = none,
frame = single,
}

\begin{lstlisting}[caption={Example of dependent tiers with word-level information}, label={lst:chatsent2}]
*MOT:   what happened ?
%mor:   pro:int|what v|happen-PAST ?
%gra:   1|2|SUBJ 2|0|ROOT 3|2|PUNCT
\end{lstlisting}

\paragraph{common tiers}
The two most commonly seen tiers are \texttt{\%mor} and \texttt{\%gra} tiers \sidenote{As shown in Listing \ref{lst:chatsent2}.}, which can be generated automatically by the use of a series of CLAN programs\todo{table: program functions} \sidenote{MOR, POST, POSTMORTEM and GRASP}. The \texttt{\%mor} tier encodes morpgological information including affixes, the stem and part-of-speech category of a word, while the \texttt{\%gra} tier keeps the dependency structure of the utterance and the grammatical relations between head and dependent words.

\newcommand{\conllu}[1]{&\footnotesize\texttt{#1}}
\newcommand{\tab}{&\hspace{0.1em}}
\setlength{\abovedisplayskip}{3pt}
\setlength{\belowdisplayskip}{3pt}
\vspace{-1em}
\section{UD \emph{CoNLL-U} Annotation Scheme}

The CoNLL-U format (\textcolor{orange}{cite Zeman? here}) is modified from the CoNLL-X format introduced by (\cite{buchholz-marsi-2006-conll}). A file in the CoNLL-U format can be seen as a collection of sentences, where each sentence is followed by an empty line. The empty lines serve as sentence boundaries.

Each CoNLL-U sentence can be divided into two parts:
\begin{itemize}
    \item comment lines \sidenote{Meta-information about the entire document is stored in the comment lines of the first sentence.} start with the hash symbol \texttt{\#} with obligatory fields \texttt{sent\_id} and \texttt{text}. \texttt{sent\_id} corresponds to a unique index in the conllu file and \texttt{text} the text of the sentence. A minimal example is shown below:
    \begin{flalign*}
    \conllu{\# sent\_id = 1}&\\
    \conllu{\# text = why ?}&
    \end{flalign*}
    \item token lines where each line represent one token or punctuation in the sentence.
\end{itemize}
\paragraph{token representation} \sidenote{For more information visit: \url{https://universaldependencies.org/format.html}}
Each token \sidenote{with the exceptions of multi-word tokens} in the text should possess one line, and each line is tab-separated into 10 fields:
\begin{enumerate}
    \item \texttt{ID:} word index starting from 1 for each new sentence \sidenote{With the exception of empty nodes, which can be a decimal number greater than 0.}.
    \item \texttt{FORM:} word form or punctuation as indicated in the sentence comment.
    \item \texttt{LEMMA:} lemma of the word
    \item \texttt{UPOS:} one of the 17 Universal part-of-speech tags \sidenote{The UPOS tags are listed in \Cref{tab:uposset} in \Cref{appendixa}}.
    \item \texttt{XPOS:} language-specific part-of-speech tags
    \item \texttt{FEATS:} a list of morphological features \sidenote{They have to be from UD's universal feature inventory or from a predefined language-specific set accepted by UD. } separated by \texttt{|}.
    \item \texttt{HEAD:} index of the head word of the current token, or \textlf{0} if the current token is the root.
    \item \texttt{DEPREL:} one of the 37 universal syntactic relations in UD \sidenote{The 37 universal syntactic relations can be found on: \url{https://universaldependencies.org/u/dep/index.html}}, or one of its subtypes.
    \item \texttt{DEPS:} head-deprel pair in the form of \texttt{head:deprel}.
    \item \texttt{MISC:} any other information related to the current token.
\end{enumerate}
\paragraph{additional constraints} The fields cannot be empty, if any information \sidenote{with the exception of word index} is not specified for a field, an underscore \texttt{\_} is used in place as a placeholder.

The tokens in the previous minimal example can be represented as follows. The top line in bold is added to indicate the field type:
\begin{align*}
\conllu{\textbf{ID}}\tab\conllu{\textbf{FORM}}\tab\conllu{\textbf{LEMMA}}\tab\conllu{\textbf{UPOS}}\tab\conllu{\textbf{XPOS}}\tab\conllu{\textbf{FEATS}}\tab\conllu{\textbf{HEAD}}\tab\conllu{\textbf{DEPREL}}\tab\conllu{\textbf{DEPS}}\tab\conllu{\textbf{MISC}}\\
\conllu{1}\tab\conllu{why}\tab\conllu{why}\tab\conllu{ADV}\tab\conllu{\_}\tab\conllu{PronType=Int}\tab\conllu{0}\tab\conllu{root}\tab\conllu{\_}\tab\conllu{\_}\\
\conllu{2}\tab\conllu{?}\tab\conllu{?}\tab\conllu{PUNCT}\tab\conllu{\_}\tab\conllu{\_}\tab\conllu{1}\tab\conllu{punct}\tab\conllu{\_}\tab\conllu{\_}
\end{align*}

\paragraph{multi-word tokens}
Multi-word tokens are treated as word groups which span multiple lines. A clear example is given on the UD website and is reproduced on the side to illustrate the point.
\marginnote{
\begin{flalign*}
\conllu{\textbf{ID}}\tab\conllu{\textbf{FORM}}\tab\conllu{\textbf{LEMMA}}\\
\conllu{1-2}\tab\conllu{v\'{a}monos}\tab\conllu{\_}\\
\conllu{1}\tab\conllu{vamos}\tab\conllu{ir}\\
\conllu{2}\tab\conllu{nos}\tab\conllu{nosotros}\\
\conllu{3-4}\tab\conllu{al}\tab\conllu{\_}\\
\conllu{3}\tab\conllu{a}\tab\conllu{a}\\
\conllu{4}\tab\conllu{el}\tab\conllu{el}\\
\conllu{4}\tab\conllu{mar}\tab\conllu{mar}
\end{flalign*}
Example taken from \url{https://universaldependencies.org/format.html}. Fields other than \texttt{ID}, \texttt{FORM} and \texttt{LEMMA} are omitted for simplicity.
}


% subsection  (end)


% \subsubsection{Morphosyntactic Coding with \emph{\%mor} Tier}
% % token links %mor to the main line. one-to-one correspondence between
% % normalised utterance and %mor.
% % structure of individual words
% % structure of word groups  -> conllu: multi-word tokens

% \subsubsection{Syntactic Dependency Analysis with \emph{\%gra} Tier}

\section{Exclusive Information vs. Shared Information}
% token as the basic unit of CHAT and CoNLL-U annotations.
% token also links %mor and %gra to the main line. one-to-one correspondence
% between normalised utterance and %mor.
Understanding how the same type of information is organised and represented in these two formats is crucial to format conversion.

As discussed above, both CHAT and CoNLL-U store meta-information as well as morphosyntactic information of tokens in the utterance. However, upon reading the manuals of both formalisms, it seems that CHAT encompasses a broader range of information types otherwise not present in CoNLL-U formats, like the required headers of a CHAT file. Therefore, although meta data in CHAT is meaningless for CoNLL-U, to reproduce CHAT from a converted file in CoNLL-U format, it must be stored in such a way that it can be put back to the right place.

Since CHAT is more encompassing than CoNLL-U and CoNLL-U files have no exclusive information that require previously non-exist tiers or headers, data contained in CoNLL-U files can be seen as a subset of that of the CHAT files.

The general information types contained in CHAT files are summarised in \Cref{tab:info}. It is clear from the table that file- and sentence-level meta data distributed in diverse places of a CHAT file are all stored as sentence comments in CoNLL-U.
\clearpage

\begin{table}[h!]
\caption {Information organisation in CHAT and CoNLL-U files} \label{tab:info}
\centering
% \begin{tabularx}{\linewidth}{@{}llXl@{}}
\begin{tabularx}{\widefigurewidth}{@{}lcXll@{}}
\toprule
\multicolumn{4}{c}{\textbf{CHAT}}& \multicolumn{1}{c}{\multirow{2}{*}{\textbf{CoNLL-U}}}\\
\cmidrule(lr){1-4}
\textbf{Level}& \textbf{Data type} & \textbf{Information} & \textbf{Storage} &\multicolumn{1}{l}{}\\
\midrule\addlinespace
\multicolumn{1}{l}{\multirow{2}{*}{\textbf{File}}}&\multicolumn{1}{c}{\multirow{2}{*}{Meta}}& obligatory headers    & file begin and end & sentence comments\\
\multicolumn{1}{l}{\multirow{2}{*}{}}&\multicolumn{1}{c}{\multirow{2}{*}{}}& other headers  & middle of the file & sentence comments\\\addlinespace
\cmidrule[0.1pt](lr{1.75em}){1-3}
\addlinespace
\multicolumn{1}{l}{\multirow{5}{*}{\textbf{Sentence}}}&\multicolumn{1}{c}{\multirow{4}{*}{Meta}}& speech codes & main line & sentence comments\\
\multicolumn{1}{l}{\multirow{5}{*}{}}&\multicolumn{1}{c}{\multirow{4}{*}{}}& speaker          & main line & sentence comments\\
\multicolumn{1}{l}{\multirow{5}{*}{}}&\multicolumn{1}{c}{\multirow{4}{*}{}}& dependent tiers  & dependent tiers & sentence comments\\
\multicolumn{1}{l}{\multirow{5}{*}{}}&\multicolumn{1}{c}{\multirow{4}{*}{}}& linkage to media  & main line & sentence comments\\\addlinespace
\multicolumn{1}{l}{\multirow{5}{*}{}}&\multicolumn{1}{c}{\multirow{1}{*}{Text}}& sentence text  & main line & sentence comments\\
\addlinespace
\cmidrule[0.1pt](lr{1.75em}){1-3}
\addlinespace
\multicolumn{1}{l}{\multirow{11}{*}{\textbf{Token}}}  &\multicolumn{1}{c}{\multirow{4}{*}{Morphological}}& word form         & main line & \texttt{FORM}\\
\multicolumn{1}{l}{\multirow{11}{*}{}}&\multicolumn{1}{c}{\multirow{4}{*}{}}& lemma           & \texttt{MOR} stem& \texttt{LEMMA}\\
\multicolumn{1}{l}{\multirow{11}{*}{}}&\multicolumn{1}{c}{\multirow{4}{*}{}}& part-of-speech  & \texttt{MOR} pos& \texttt{UPOS} and \texttt{XPOS}\\
\multicolumn{1}{l}{\multirow{11}{*}{}}&\multicolumn{1}{c}{\multirow{4}{*}{}}& features        & \texttt{MOR} affixes& \texttt{FEATS}\\\addlinespace
\multicolumn{1}{l}{\multirow{11}{*}{}}&\multicolumn{1}{c}{\multirow{3}{*}{Syntactic}}& word index      & \texttt{GRA} index& \texttt{ID}\\
\multicolumn{1}{l}{\multirow{11}{*}{}}&\multicolumn{1}{c}{\multirow{3}{*}{}}& head            & \texttt{GRA} head& \texttt{HEAD}\\
\multicolumn{1}{l}{\multirow{11}{*}{}}&\multicolumn{1}{c}{\multirow{3}{*}{}}& syntactic relation  & \texttt{GRA} GR label& \texttt{DEPREL}\\\addlinespace
\multicolumn{1}{l}{\multirow{11}{*}{}}&\multicolumn{1}{c}{\multirow{4}{*}{Additional}}& clitic type      & \texttt{MOR} & \texttt{MISC}\\
\multicolumn{1}{l}{\multirow{11}{*}{}}&\multicolumn{1}{c}{\multirow{4}{*}{}}& compound components  & \texttt{MOR} & \texttt{MISC}\\
\multicolumn{1}{l}{\multirow{11}{*}{}}&\multicolumn{1}{c}{\multirow{4}{*}{}}& special forms     & main line & \texttt{MISC}\\
\multicolumn{1}{l}{\multirow{11}{*}{}}&\multicolumn{1}{c}{\multirow{4}{*}{}}& ...              & ... & \texttt{MISC}\\\addlinespace
\bottomrule
\end{tabularx}
\end{table}

As for the shared morphosyntactic information, while CHAT uses dependent tiers parallel to the main line, UD takes a token-based approach. Nevertheless, both formalisms use word as the basic unit for morphosyntactic analyses. Thus, although placed in separate tiers, morphological information in the \texttt{\%mor} tier and syntactic relations between words in the \texttt{\%gra} tier can be associated with tokens in the utterance. Similarly, by combining information separated in different fields of the CoNLL-U token line, one can group and reorganise the fields to produce string representations similar to those assigned by MOR and GRASP.
