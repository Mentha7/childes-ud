\chapter{CHILDES \emph{CHAT} and UD \emph{CoNLL-U}}
\label{Chapter2}

This chapter introduces the two annotation schemes of interest---\emph{CHAT} and \emph{CoNLL-U}---in detail and gives the reader an idea of what information is shared by both annotations and what information is exclusively relevant to only one of them. This will shed light on the implementation of \emph{chatconllu} which will be discussed in \Cref{Chapter3}.

\section{CHILDES CHAT Transcription System}

\subsection{Motivation}
The \emph{Codes for Human Analysis of Transcripts} (CHAT) transcription system is developed for the (\cite{Macwhinney2000})\emph{Child Language Data Exchange System} (CHILDES) project (\cite{Macwhinney2000}) to study child language acquisition.

As MacWhinney pointed out in the CHAT manual, collecting audio data for spontaneous conversational interactions is easy, one just have to turn on the recorder, however, processing the collected recordings and turning them into usable data for linguistic research is a demanding task (\cite{Macwhinney2000}). Transcribing the conversations is such an indispensable preprocessing step. Consistency treatment of raw data is critical to scientific research. Therefore, it is hard to imagine how difficult it was to do consistent transcription when there was no standard format, especially for child speech, which is usually more chaotic than average adult speech data. Thus CHAT is designed to accommodate the need for a standard transcription system of conversational speech. Moreover, CHAT transcripts can be analysed by TalkBank programs \sidenote{\url{https://talkbank.org/software/}} like \emph{Computerized Language Analysis} (commonly referred to as the CLAN Program), which facilitates the transcription, sharing and analysis of human conversational interactions.

\subsection{File Structure}
For a CHAT file to be well-formed, there are some basic structural requirements to comply with. The minimum standards are prescribed by \emph{minCHAT} (\cite{Macwhinney2000}).

% describe only relevant minCHAT structure
First of all, each CHAT file should have at least two parts:
\begin{itemize}
	\item \emph{headers} preceded by the \texttt{@} symbol that stores metadata like language and participants. Some headers are required, some are optional.\\
	Despite its name, headers can be added in the middle of a CHAT file to represent information like tape locations. It is likely that \emph{header} is a misnomer, and they should be called \emph{meta} as they preserve in general higher-level information.
	\item the \emph{main line}---utterances led by the asterisk symbol \texttt{*} which are the actual sentences being said.\\
	Additional information can be \emph{optionally} added as dependent tiers. Annotations of morphological, syntactical, and phonological information or general remarks of the transcriber's choice all go into separate dependent tiers of the main utterance. Dependent tiers are preceded by the percent symbol \texttt{\%}.
\end{itemize}

% describe mor and gra tiers
\subsection{Dependent Tiers}
Almost any type of information can be organised into separate dependent tiers and added to each utterance. The CHAT manual defines a standard list of dependent tiers. The tier names are mostly 3-lowercase-letter codes, but with the exceptions of tier names preceded by an \texttt{x}, like \texttt{\%xmor}.

A dependent tier can be organised in one of two ways:
\begin{itemize}
	\item utterance-level information
	\item word-level information
\end{itemize}

\paragraph{utterance-level}
Utterance-level information usually appear as a descriptive sentence. For example, the action tier that describe actions by the speakers \texttt{\%act}. It can also be in other forms, like the addressee tier which uses three-letter participant codes.

\paragraph{word-level}
Word-level information are associated with words in the utterance. Pieces of word-based information string are separated by the space character.

\paragraph{common tiers}
The two most commonly seen tiers are \texttt{\%mor} and \texttt{\%gra} which can be generated automatically by the use of a series of CLAN programs---MOR, POST, POSTMORTEM and GRASP.

The \texttt{\%mor} tier encodes morpgological information including the affixes, stem and part-of-speech tags of words, while the \texttt{\%gra} tier keeps the dependency structure of the utterance and the grammatical relations between head and dependent words.

Since morphosyntactic information in \texttt{\%mor} and \texttt{\%gra} should be kept with individual tokens in CoNLL-U files, they have to be processed in such a way that all relevant information from both tiers (if present) are associated with the respective token in the utterance.

% \subsubsection{Morphosyntactic Coding with \emph{\%mor} Tier}
% % token links %mor to the main line. one-to-one correspondence between
% % normalised utterance and %mor.
% % structure of individual words
% % structure of word groups  -> conllu: multi-word tokens

% \subsubsection{Syntactic Dependency Analysis with \emph{\%gra} Tier}

\section{UD \emph{CoNLL-U} Annotation Scheme}

\subsection{Motivation}

\subsection{File Structure} % (fold)
\label{sub:File Structure}

\subsection{UD Sentence}

\subsection{UD Token}

\paragraph{multi-word tokens}

% subsection  (end)

\section{Shared Information vs. Exclusive Information}
% token as the basic unit of CHAT and CoNLL-U annotations.
% token also links %mor and %gra to the main line. one-to-one correspondence
% between normalised utterance and %mor.
