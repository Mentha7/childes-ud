\chapter{Validation and Comparison} % Main chapter title

\label{Chapter5} % For referencing the chapter elsewhere, use \ref{Chapter5}

\section{Validating CHAT using CLAN Tools}

After using chatconllu to convert the conllu files back to CHAT format, the user may want to validate that the resulting cha files are well-formed and accepted by the CLAN program. CLAN has a CHECK program (include citation here) that validates the CHAT transcription format. Alternatively, if the user would like to stick to the command line, he/she can download Chatter (include citation here) which is a Java program from TalkBank. 

% include a use case.

For the chosen corpora, cha files produced by chatconllu pass the CHECK program in CLAN but not Chatter for the reason of line wrapping. Long lines in the original cha files generated from CLAN program are broken after a certain amount of characters is reached. Continuations of the line are preceded by an initial tab. The current version of chatconllu, however, does not handle the maximum length of a line and because of that, it fails the Chatter validation process.

The differences between the original and back-converted cha files can be visually inspected using \emph{icdiff}. A side-by-side display of the two cha files selected is displayed in the terminal, with the differences coloured. 

\section{Validating CoNLL-U using UD Tools}

UD also provides tools for format validation. One of the UD maintained tools is the script \emph{validate.py} (include reference here). However, what's tricky about UD format validation is that this script validate also the accepted values of each language in UD. To run the script, one has to input a language code. There are 5 levels of validation, right now the conllu files generated by chatconllu only pass the first level which does not check if the values are accepted by UD or specifications of the given language.