\chapter{Validation and Comparison} % Main chapter title

\label{Chapter5} % For referencing the chapter elsewhere, use \ref{Chapter5}

\section{Validating CHAT using CLAN Tools}

After using chatconllu to convert the CoNLL-U files back to CHAT format, the user may want to validate that the resulting CHAT files are well-formed and accepted by the CLAN program. CLAN has a CHECK program \sidenote{CLAN can be downloaded via \url{https://dali.talkbank.org/clan/}} that can validate the CHAT transcription format. Alternatively, if the user wants to stick to the command line, they can download Chatter \sidenote{Chatter - CHAT format validator: \url{https://talkbank.org/software/chatter.html}}, a Java program from TalkBank, and follow the instructions on the website.\\

For the chosen corpora, CHAT files produced by chatconllu pass the CHECK program in CLAN but not Chatter for the reason of missing tabs. Long lines in the original cha files generated from the CLAN program are broken into separate lines after a certain amount of characters is reached. Continuations of the line are preceded by an initial tab. However, the current version of chatconllu does not handle the maximum length of a line, and because of that, it fails the Chatter validation process.
\vspace{-0.4em}
\section{Validating CoNLL-U using UD Tools}

UD also provides tools for format validation. One of the UD maintained tools is the script \emph{validate.py} \sidenote{link to file on GitHub: \url{https://github.com/UniversalDependencies/tools/blob/master/validate.py}}. Its validation is in five levels. To run the script, one has to input a language code because in high-level tests, this script also check whether the values in the fields are allowed for that language. Currently, chatconllu only pass the first two levels at best, which does not check if the values are accepted by UD or specifications of the given language.
\vspace{-0.4em}
\section{Comparison with Original Files}
The differences between the original and back-converted cha files can be visually inspected using terminal tools like \emph{icdiff} \sidenote{icdiff: \url{https://www.jefftk.com/icdiff}}. A side-by-side view of the two cha files selected is displayed in the terminal, with the differences coloured. As mentioned above, most visible differences are the results of the missing leading tabs. Unfortunately, I haven't figured out a way to solve this problem programmatically.
