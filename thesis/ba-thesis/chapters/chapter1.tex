\chapter{Introduction} % Main chapter title

\label{Chapter1} % For referencing the chapter elsewhere, use \ref{Chapter1}

% intro
Traditionally, linguistic research has primarily been language-specific, with theories stem from studies on Indo-European languages. In recent decades, there has been much progress in crosslinguistic research, one of which is the \emph{universal dependencies} (UD) project (\cite{nivre2016} and \cite{nivre2020}).

As Joakim Nivre explained in his chapter \emph{Towards a Universal Grammar in Natural Language Processing}, instead of a descriptive and explanatory account of morphosyntax, UD's universal grammar is developed for natural language processing applications(\cite{nivre2015}). Backed by both universal and language-specific tagsets, UD's \emph{CoNLL-U} annotation allows for crosslinguistic corpus sharing. However, most UD treebanks are still based on written text.

Like CoNLL-U, \emph{CHAT} is a standardised annotation scheme. It is part of the \emph{Child Language Data Exchange System} (CHILDES) project (\cite{Macwhinney2000}), and it offers a standard way to transcribe spontaneous speech. Morphosyntactic information can also be added as dependent tiers to the main utterances, and if preferred, it can be generated automatically by running \emph{MOR} (\cite{Macwhinney2000}), \emph{POST} (\cite{parisse2000}) and \emph{GRASP} (\cite{Sagae2004}).

It would be beneficial to use the advantages of both formalisms to compensate each other---CHILDES can provide UD with spontaneous speech treebanks, and UD can offer its tagsets which are widely accepted. To achieve this goal, a two-way converter, chatconllu, is is designed and implemented as a command-line tool in this project.

% explain the choice of the 2 annotations first
The aim of this thesis project is to explore and experiment with format conversion between the above-mentioned annotation schemes via implementing a two-way converter in Python. The scope of analysis is limited to a selected few corpora from the CHILDES database, listed in \Cref{tab:martabdb}. Transcriptions in CHAT format are first converted to CoNLL-U format and then converted back to see which information is lost in the process. With the aid of tools like UDPipe \cite{straka-etal-2016-udpipe}, new information is augmented to the corpora and after chatconllu's conversion, is represented in CHAT-style dependent tiers. To verify the converted and back-converted files, both formats are validated using the tools provided by the two projects---CHILDES and UD.This thesis also serves as a complement to the code on GitHub \sidenote{childes-ud/chatconllu: \url{https://github.com/Mentha7/childes-ud}}.

The thesis is structured as follows: \Cref{Chapter2} gives necessary background information of both annotation schemes. \Cref{Chapter3} introduces the converter and its implementation. \Cref{Chapter4} discusses the mappings from CHILDES to UD. \Cref{Chapter5} explains the validation process and results, outlines related work and propose future improvements for chatconllu. Finally, with \Cref{Chapter6} I conclude this thesis and outlines related work.

