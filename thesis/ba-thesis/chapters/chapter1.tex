\chapter{Introduction} % Main chapter title

\label{Chapter1} % For referencing the chapter elsewhere, use \ref{Chapter1}

% intro
Traditionally, linguistic research has primarily been language-specific, with theories stem from studies on Indo-European languages. In recent decades, there has been much progress in crosslinguistic research, one of which is the Universal Dependencies (UD) Project (\cite{nivre2016}, \cite{nivre2020}). As Joakim Nivre explained in his chapter \emph{Towards a Universal Grammar in Natural Language Processing} , instead of a descriptive and explanatory account of morphosyntax, UD's universal grammar is devised for natural language processing (\cite{nivre2015}). Backed by both universal and language-specific tagsets, UD's CoNLL-U annotation is a scheme that allows for crosslinguistic corpus sharing. However, most UD treebanks are still based on written text.

Like CoNLL-U, CHAT (Codes for Human Analysis of Transcripts) (\cite{Macwhinney2000}) is a standardised annotation scheme. It is part of the CHILDES (Child Language Data Exchange System) Project (\cite{Macwhinney2000}), and it offers a standard way to transcribe spontaneous speech. Morphosyntactic information can also be added as dependent tiers to the main utterances, and if preferred, automatically by running MOR (\cite{Macwhinney2000}), POST (\cite{parisse2000}) and GRASP (\cite{Sagae2004}).

It would be interesting to use the advantages of both formalisms to compensate each other. With the help of a converter, CHILDES can provide UD with spontaneous speech treebanks, and UD can offer its tagsets which are widely accepted. There are also many tools developed to process CoNLL-U formatted files, with which information can be augmented to the CHAT files. It is based on this motivation that chatconllu is designed and implemented.

% explain the choice of the 2 annotations first
This thesis aims to explore and experiment with format conversion between these two annotation schemes via designing and implementing a two-way converter with Python. The scope of analysis is limited to a selected few corpora from the CHILDES database. Transcriptions in \emph{CHAT} are first converted to \emph{CoNLL-U} and then converted back to see which information is lost in the process. Both formats are validated using the tools provided on their official websites.

This thesis serves as a complement to the code on GitHub \sidenote{childes-ud: \url{https://github.com/Mentha7/childes-ud}}. \Cref{Chapter2} gives necessary background information of both annotation schemes and outlines related work. \Cref{Chapter3} introduces the converter and its implementation. \Cref{Chapter4} discusses the mappings from CHILDES to UD. \Cref{Chapter5} explains the validation process and discusses the current weaknesses and future improvements of chatconllu. Finally, with \Cref{Chapter6} I conclude this thesis.



% explain the choice of the 2 annotations, to be finished separately and integrated into the section above.
% \section{CHILDES}

% The Child Language Data Exchange System (CHILDES) is part of TalkBank project with focus on child language.

% \section{UD}

