\chapter{Ideas} % Main chapter title

\label{Chapter8} % For referencing the chapter elsewhere, use \ref{Chapter8}

\section{Morphosyntax in CHAT and CoNLL-U}

Both annotation formalisms support the storage of morphosyntactic information, although in different ways. In this chapter, I show the general structure of these formats using minimal examples, compare their ways of organising information and point out issues that I found worth discussing during the implementation process of chatconllu.

\subsection{Extracting Information from Dependent Tiers}



\paragraph{lemma} After looking at the syntax of the \%mor tier, one should realise that getting the lemma out of a MOR string is not an easy task. There are several pitfalls to avoid:\\
\begin{itemize}
	\item The semantics of the symbols may be ambiguous. Take the dash symbol as an example, for dashed words (words with dash symbols) like \texttt{tic-tac-toe}, it is merely a connector between parts of the word, but it could also appear as an indicator for suffixes, as in the example of \texttt{part|get-PRESP} for the word \emph{getting}.
	\item One should also notice that the stem is not the lemma, just that for English, these two share the same form in most cases. For instance, for \emph{untied}, we have the MOR segment \texttt{un\#v|tie-PASTP}, which means in plain English that the word has a stem \emph{tie} with prefix \emph{un}, its part-of-speech is verb and this word form is the past participle of the verb. On the other hand, in CoNLL-U, the lemma field is reserved for lemma, not word stem, therefore the prefix should be put back in place to produce the real lemma.
\end{itemize}


\texttt{prefix\#pos|stem\&fusionalsuffix-suffix=translation}

\texttt{compound_pos|+component_pos|component_stem+component_pos|component_stem}

\texttt{prefix\#pos|stem\&fusionalsuffix-suffix=translation}


\section{Related Work}

\paragraph{pylangacq} % (fold)
\label{par:pylangacq}

\paragraph{pyconll} % (fold)
\label{par:pyconll}

\paragraph{CHAT2CONLLU Program} % (fold)
\label{par:chat2conllu}

\paragraph{Zoey Liu's Project} % (fold)
\label{par:zoey}
