\chapter{Ideas} % Main chapter title

\label{Chapter8} % For referencing the chapter elsewhere, use \ref{Chapter8}

\section{Morphosyntax in CHAT and CoNLL-U}

Both annotation formalisms support the storage of morphosyntactic information, although in different ways. In this chapter, I show the general structure of these formats using minimal examples, compare their ways of organising information and point out issues that I found worth discussing during the implementation process of chatconllu.

%?
There are also many tools developed to process CoNLL-U formatted files, like UDPipe \cite{straka-etal-2016-udpipe}, with which additional information can be augmented to the CHAT files. As a format converter, chatconllu strives to serve as an initial (and sometimes final) procesisng step in this pipeline of information augmentation via the other format. Moreover, chatconllu can generate CHAT-style dependent tiers to represent augmented information. By comparing the augmented information with the original annotations, one can compare the performance of different parsers.


% \begin{table}[h!]
% \caption {Information organisation in CHAT and CoNLL-U files} \label{tab:grmap}
% \centering
% % \begin{tabularx}{\linewidth}{@{}llXl@{}}
% \begin{tabularx}{\widefigurewidth}{@{}lcXX@{}}
% \toprule
% \textbf{Info level}          & \textbf{Data type} & \textbf{CHAT} & \textbf{CoNLL-U}\\ \midrule
% \textbf{File}           & meta          & file-initial and file-final headers & sentence comments\\
% 						&               & other headers & sentence comments\\\addlinespace\addlinespace
% \textbf{Sentence}       & meta          & codes for speech errors, etc.& sentence comments\\
% 						&               & speaker & sentence comments\\
% 						&               & arbitrarily many dependent tiers & sentence comments\\
% 						&               & linkage to media & sentence comments\\\addlinespace\addlinespace
% \textbf{Token}          & morphological & word form (in utterance)& \texttt{FORM}\\
% 						&               & lemma (part of \texttt{MOR} code)& \texttt{LEMMA}\\
% 						&               & part-of-speech (part of \texttt{MOR} code)& \texttt{UPOS} and \texttt{XPOS}\\
% 						&               & morphological features (part of \texttt{MOR} code)& \texttt{FEATS}\\\addlinespace
% 						& syntactic     & word index (part of \texttt{GRA} code)& \texttt{ID}\\
% 						&               & head (part of \texttt{GRA} code)& \texttt{HEAD}\\
% 						&               & syntactic relation (part of \texttt{GRA} code)& \texttt{DEPREL}\\\addlinespace
% 						& additional    & clitic type (part of \texttt{MOR} code)& \texttt{MISC}\\
% 						&               & compound components (part of \texttt{MOR} code)& \texttt{MISC}\\
% 						&               & special word forms (in utterance)& \texttt{MISC}\\
% 						&               & ... & \texttt{MISC}\\\bottomrule
% \end{tabularx}
% \end{table}


% \begin{table}[h!]
% \caption {Information organisation in CHAT and CoNLL-U files} \label{tab:info}
% \centering
% % \begin{tabularx}{\linewidth}{@{}llXl@{}}
% \begin{tabularx}{\linewidth}{@{}Xll@{}}
% \toprule
% \multicolumn{2}{c}{\textbf{CHAT}}& \multicolumn{1}{c}{\multirow{2}{*}{\textbf{CoNLL-U}}}\\
% \cmidrule(lr){1-2}
% Information & Storage &\multicolumn{1}{l}{}\\\midrule
% lemma & \texttt{MOR}& \texttt{LEMMA} \\
% part-of-speech & \texttt{MOR}& \texttt{UPOS} and \texttt{XPOS}\\
% features & \texttt{MOR}& \texttt{FEATS} \\\addlinespace
% clitic type & \texttt{MOR} & \texttt{MISC} \\
% compound components & \texttt{MOR} & \texttt{MISC} \\
% translation & \texttt{MOR} & \texttt{MISC} \\
% prefix & \texttt{MOR} & \texttt{MISC} \\\bottomrule
% \end{tabularx}
% \end{table}


% \begin{table}[h!]
% \caption {Information organisation in CHAT and CoNLL-U files} \label{tab:info}
% \centering
% % \begin{tabularx}{\linewidth}{@{}llXl@{}}
% \begin{tabularx}{\linewidth}{@{}Xll@{}}
% \toprule
% \textbf{MOR} & \textbf{Field}\\\midrule
% lemma & \texttt{LEMMA} \\
% part-of-speech & \texttt{UPOS}, \texttt{XPOS}\\
% features & \texttt{FEATS} \\\addlinespace
% clitic type & \texttt{MISC} \\
% compound components & \texttt{MISC} \\
% translation & \texttt{MISC} \\
% prefix & \texttt{MISC} \\\bottomrule
% \end{tabularx}
% \end{table}


% \begin{table}[h!]
% \caption {Information organisation in CHAT and CoNLL-U files} \label{tab:info}
% \centering
% % \begin{tabularx}{\linewidth}{@{}llXl@{}}
% \begin{tabularx}{\linewidth}{@{}Xll@{}}
% \toprule
% \textbf{GRA} & \textbf{Field}\\\midrule
% word index & \texttt{ID} \\
% head & \texttt{HEAD}\\
% syntactic relation & \texttt{DEPREL} \\\bottomrule
% \end{tabularx}
% \end{table}


% \section{Dependency Relation Mapping}\todo{change the table}
% \begin{table}[h!]
% \caption {chatconllu Dependency Relation Mapping} \label{tab:grmap}
% \centering
% \begin{tabularx}{\linewidth}{@{}lXl@{}}
% \toprule
% \textbf{Relation type} & \textbf{GRA grammatical relation label} & \textbf{description} & \textbf{UD deprel}\\ \midrule
% 	Predicate-head & SUBJ& non-clausal subject & nsubj\\
% 	& CSUBJ& clausal subject & csubj\\
% 	& OBJ& first or direct object & obj\\
% 	& OBJ2& second or indirect object & iobj\\
% 	& COMP& complement clause & ccomp\\
% 	& PRED& nominal or adjectival predicate & cop or xcomp\\
% 	& CPOBJ& clausal prepositional object & \\
% 	& COBJ& clausal object & \\
% 	& POBJ& non-clausal prepositional object & \\
% 	& SRL& serial verb construction & \\
% 	& JCT& adjunct that modifies a verb, adjective or adverb& \\
% 	& CJCT& clausal conjunct & \\
% 	& XJCT& non-finite clausal adjunct & \\
% 	& NJCT& adjunctthat modifies a noun & \\
% 	& MOD& non-clausal modifier & \\
% 	& POSTMOD& postposed nominal modifier & \\
% 	& POSS& possessive & \\
% 	& APP& appositive & \\
% 	& CMOD& clausal modifier & \\
% 	& XMOD& Tense=Past & \\
% 	& DET& determiner & \\
% 	& QUANT& Number=Plur & \\
% 	& PQ& poss Poss=Yes & \\
% 	& AUX& auxiliary & \\
% 	& NEG& verbal negator & \\
% 	& INF& Tense=Pres & \\
% 	& LINK& Tense=Pres|VerbForm=Part & \\
% 	& TAG& Tense=Past & \\
% 	& COM& Number=Sing & \\
% 	& BEG& Degree=Sup & \\
% 	& END& Tense=Past|VerbForm=Part & \\
% 	& INCROOT& Number=Plur & \\
% 	& OM& poss Poss=Yes & \\
% 	& PUNCT& punctuation & \\
% 	& LP& local punctuation & \\
% 	& BEGP&  & \\
% 	& ENDP&  & \\
% 	& ROOT& root & \\
% 	& NAME& Degree=Sup & \\
% 	& DATE& Mood=Sub & \\
% 	& ENUMERATION& series of elements not linked through  & \\
% 	& CONJ& link from  & \\
% 	& COORD& Degree=Sup & \\
% 	& BEGP& Mood=Sub & \\
% 	& END& Mood=Sub & \\\bottomrule
% \end{tabularx}\\
% \vspace{0.5cm}
% All GRASP GRs handled by the current version of chatconllu.\\
% \end{table}
