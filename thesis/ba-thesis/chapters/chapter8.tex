\chapter{Ideas} % Main chapter title

\label{Chapter8} % For referencing the chapter elsewhere, use \ref{Chapter8}

\section{Morphosyntax in CHAT and CoNLL-U}

Both annotation formalisms support the storage of morphosyntactic information, although in different ways. In this chapter, I show the general structure of these formats using minimal examples, compare their ways of organising information and point out issues that I found worth discussing during the implementation process of chatconllu.

%?
There are also many tools developed to process CoNLL-U formatted files, like UDPipe \cite{straka-etal-2016-udpipe}, with which additional information can be augmented to the CHAT files. As a format converter, chatconllu strives to serve as an initial (and sometimes final) procesisng step in this pipeline of information augmentation via the other format. Moreover, chatconllu can generate CHAT-style dependent tiers to represent augmented information. By comparing the augmented information with the original annotations, one can compare the performance of different parsers.


\begin{table}[h!]
\caption {Information organisation in CHAT and CoNLL-U files} \label{tab:grmap}
\centering
% \begin{tabularx}{\linewidth}{@{}llXl@{}}
\begin{tabularx}{\widefigurewidth}{@{}lcXX@{}}
\toprule
\textbf{Info level}          & \textbf{Data type} & \textbf{CHAT} & \textbf{CoNLL-U}\\ \midrule
\textbf{File}           & meta          & file-initial and file-final headers & sentence comments\\
                        &               & other headers & sentence comments\\\addlinespace\addlinespace
\textbf{Sentence}       & meta          & codes for speech errors, etc.& sentence comments\\
                        &               & speaker & sentence comments\\
                        &               & arbitrarily many dependent tiers & sentence comments\\
                        &               & linkage to media & sentence comments\\\addlinespace\addlinespace
\textbf{Token}          & morphological & word form (in utterance)& \texttt{FORM}\\
                        &               & lemma (part of \texttt{MOR} code)& \texttt{LEMMA}\\
                        &               & part-of-speech (part of \texttt{MOR} code)& \texttt{UPOS} and \texttt{XPOS}\\
                        &               & morphological features (part of \texttt{MOR} code)& \texttt{FEATS}\\\addlinespace
                        & syntactic     & word index (part of \texttt{GRA} code)& \texttt{ID}\\
                        &               & head (part of \texttt{GRA} code)& \texttt{HEAD}\\
                        &               & syntactic relation (part of \texttt{GRA} code)& \texttt{DEPREL}\\\addlinespace
                        & additional    & clitic type (part of \texttt{MOR} code)& \texttt{MISC}\\
                        &               & compound components (part of \texttt{MOR} code)& \texttt{MISC}\\
                        &               & special word forms (in utterance)& \texttt{MISC}\\
                        &               & ... & \texttt{MISC}\\\bottomrule
\end{tabularx}
\end{table}

