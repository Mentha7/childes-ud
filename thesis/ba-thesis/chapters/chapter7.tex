\chapter{Deterministic Mapping for Morphosyntax} % Main chapter title
\label{Chapter7} % For referencing the chapter elsewhere, use \ref{Chapter7}

In the previous chapter, the different ways of how CHAT and CoNLL-U structurally represent morphosyntactic information are explained. This chapter presents a deterministic mapping from CHAT codes to UD-style labels. The mapping is created for English, French and Italian and consists of three major categories: part-of-speech tags, morphological features and dependency relations.

% design differences

% converting MOR codes to UD Universal POS tags
\section{Part-of-Speech Tags}
\label{sec:pos}
% left-aligned table columns with automated line breaks
\newcolumntype{L}{>{\RaggedRight\arraybackslash}X}
\begin{margintable}[1\baselineskip] % move figure down by 1 line
\begin{tabularx}{\textwidth}{@{}Ll@{}}
\toprule
\textbf{MOR} & \textbf{UD}\\ \midrule
adj & ADJ\\
adv, neg & ADV\\
co, fil & INTJ\\
qn, det & DET\\
aux, cop, mod & AUX\\
post, prep & ADP\\
part, inf & PART\\
coord & CCONJ\\
comp & SCONJ\\
num & NUM\\
n & NOUN\\
pro & PRON\\
n:prop & PROPN\\
v & VERB\\
end, beg & PUNCT\\
none, dia & X\\\bottomrule
\end{tabularx}
\caption{\label{tab:martabpos}Examples of MOR POS categories that can be directly converted to UPOS tags. See Table \ref{tab:posmap1} and Table \ref{tab:posmap2} in \Cref{appendixa} for the full mapping.}
\footnotesize
% Examples of MOR POS categories that can be directly converted to UPOS tags. See \ref{tab:posmap1} and \ref{tab:posmap2} in \Cref{appendixa} for the full mapping.
\end{margintable}


According to the CHAT manual, the MOR program, which deals with part-of-speech tagging, has 39 POS tags for English with extensions for other languages, while UD uses a universal POS (UPOS) tagset with 17 tags for all languages in the Universal Dependencies Project (as of version 2.8).

UD's UPOS is developed based on Petrov's \emph{A Universal Part-of-Speech Tagset}, where the crosslinguistic universality of the tagset stems from its operational definition instead of an intrinsic definition of the grammatical categories (\cite{petrov2012}). In CHAT, however, POS tags are much less universal across languages. As described by the CHAT Manual (Section 2.4 Part of Speech Codes), typologically different languages may use quite different tagsets than that prescribed by the MOR program for English. Additionally, researchers can define their own POS tagsets (\cite{Macwhinney2000}), which not only means that in terms of POS tags, CHILDES took an opposite approach from UD, but also makes an exhaustive mapping between POS tags used in CHILDES databases and the UPOS tagset impossible, even just for the three languages chosen. Therefore, the mapping I define here concerns only the standard POS tagset given by the MOR grammar for English, French and Italian.

The adaptation from MOR POS categories to UD's UPOS can be achieved in the following ways:
\begin{itemize}
    \item a direct mapping where the two tags are the same or has a straightforward relationship
    \item collapsing many MOR POS categories into one UPOS tag when applicable
    \item best-effort decision-making based on the description of the POS cetegories by both formalisms
    \item MOR POS categories with no UPOS counterparts are translated as \texttt{X}
\end{itemize}

An example of straightforward mappings between MOR POS codes and UPOS tags is shown in Table \ref{tab:martabpos}. A total of 16 out of the 17 UPOS tags are used in the tagset mapping, with the exception of \texttt{SYM}, which is reserved for symbols.

Collapsing of POS categories can be illustrated by the example of auxiliary verbs. The MOR code differentiates between auxiliary verbs \texttt{aux} or \texttt{v:aux}, copulas \texttt{cop} and modal verbs \texttt{mod}. However, since UD took an operational approach to define grammatical categories, it does not distinguish copulas or modal verbs from other verbs that function as auxiliaries.  In doing so, UPOS can be applied to typologically distant languages.

The following MOR codes require more thoughts during tagset conversion, partially because they do not exist in all languages. For tags like these, UD usually raises them to a more coarse category and specifies the exact type as morphological features.

\paragraph{\texttt{art}}
Articles are ganerally marked as \texttt{DET} in UD Treebanks with feature \texttt{PronType=Art} \sidenote{Definition for \texttt{DET} can be found from \url{https://universaldependencies.org/docs/u/pos/DET.html}}.

\paragraph{\texttt{preart}}
\texttt{preart} are contractions of prepositions and articles, for instance the Italian \emph{alla} is the contraction of \emph{a} and \emph{la}. In UD-style annotations, words like \emph{alla} are treated as multi-word tokens, and a POS tag is given to each part of the token. However, CHAT considers them to be single words with forms of their own. To keep the alignment with \%mor tier, here \texttt{preart} is mapped to \texttt{DET} because the word functions as a determiner.

\paragraph{\texttt{vimp}}
Imperatives are tagged with \texttt{vimp} in \%mor tier. However, in UD, imperatives are tagged with \texttt{VERB}, which is the category for all finite verbforms \sidenote{Definition for \texttt{VERB} can be found from \url{https://universaldependencies.org/docs/u/pos/VERB.html}} . Information of gender, person, number, tense, aspect, mood and voice are all specified as features. Therefore \texttt{vimp}, as an example, is mapped to \texttt{VERB} with feature \texttt{MOOD=Imp}.

\paragraph{\texttt{vpfx}}
Preverb/verb particles are tagged with \texttt{vpfx}. Although they are particles, according to UD's definition \sidenote{Definition for \texttt{ADP} can be found from \url{https://universaldependencies.org/docs/u/pos/ADP.html}}, they belong to \texttt{ADP}.

\paragraph{\texttt{conj}}
Despite the similarity between the MOR \texttt{conj} and UD \texttt{CCONJ}, they represent different things. UD distinguishes between \texttt{CCONJ} \sidenote{Definition for \texttt{CCONJ} can be found from \url{https://universaldependencies.org/docs/u/pos/CCONJ.html}} and \texttt{SCONJ} \sidenote{Definition for \texttt{SCONJ} can be found from \url{https://universaldependencies.org/docs/u/pos/SCONJ.html}}, the former is for coordinating conjunctions like \emph{and} and \emph{but}, which are tagged \texttt{coord} by MOR, and the latter marks subordinating conjunctions with complementizers like \emph{that} or \emph{if}, or adverbial clause introducers like \emph{when}, these are tagged \texttt{conj} by MOR. Therefore, the UPOS tag corresponding to \texttt{conj} is \texttt{SCONJ}.

\paragraph{special form markers}
It is worth noticing that some MOR codes do not correspond to any grammatical categories. For example, in CHAT, the so-called \emph{special form markers} have unique POS tags independent of the part-of-speech category to which the word belongs. Nevertheless, some of these tags can still be mapped to UPOS, like \texttt{sing} for singing, \texttt{bab} for babbling and \texttt{wplay} for wordplay can all be mapped to \texttt{INTJ}, which stands for \emph{interjection}. The definition of interjection, however, is vague. UD defines interjection descriptively \sidenote{"An interjection is a word that is used most often as an exclamation or part of an exclamation. It typically expresses an emotional reaction, is not syntactically related to other accompanying expressions, and may include a combination of sounds not otherwise found in the language." Excerpt from \url{https://universaldependencies.org/docs/u/pos/SCONJ.html}}. The core criteria for a word \sidenote{or non-word in this case} to be an interjection include:
\begin{itemize}
    \item that it is often used exclamatorily
    \item that it typically expresses emotions
    \item that it plays no syntactic role in the utterance
    \item that it might not be a recognised word in the language
\end{itemize}
Since singing, babbling and wordplay satisfy these criteria, along with the MOR code explicitly for interjections \texttt{co}, they are granted the \texttt{INTJ} tag. Other special form markers, on the other hand, has no corresponding UPOS tags whatsoever, for example \texttt{dia} for dialect forms. Unfortunately they have to be assigned to \texttt{X}, for words that does not belong to the other 16 categories.

\begin{margintable}[1\baselineskip] % move figure down by 1 line
\begin{tabularx}{0.85\textwidth}{@{}cl@{}}
\toprule
Punctuations & MOR Code\\\midrule
„ & end\\
‡ & beg\\
, & cm\\
“ & bq\\
” & eq\\
‘ & bq2\\
’ & eq2\\\bottomrule
\end{tabularx}
\caption{\label{tab:martabpunct}MOR punctuation marks and their corresponding MOR codes.}
\footnotesize
\end{margintable}

\paragraph{punctuation marks}
Finally, MOR also defines a set of punctuation marks with string representations, shown in Table \ref{tab:martabpunct}. Although treated as lexical items in the \%mor tier, they are all given the \texttt{PUNCT} tag.

% feature mappings
\section{Morphological Features}
\label{sec:feats}

Morphological features like \emph{tense} and \emph{mood} are encoded in the string representation of the word. MOR uses a morpheme-based approach to analyse individual words. As discussed in the previous chapter, a word is separated into prefix, part-of-speech, stem, affixes or morphemes indicating specific morphological features. Unlike MOR's dissecting analysis of the word, UD treats words as the basic unit, and features are considered properties of the word. It does not presuppose the existence of morphemes.

\begin{margintable}[1\baselineskip]
\begin{tabularx}{1\textwidth}{@{}ll@{}}
\toprule
\textbf{CHILDES Database} & \textbf{Language}\\ \midrule
    Brown & eng\\
    English-MiamiBiling & eng\\
    Belfast & eng\\
    Wells & eng\\
    MTLN & fra\\
    Leveille & fra\\
    vioncolas & fra\\
    Geneva & fra\\
    Tonelli & ita\\\bottomrule
\end{tabularx}
\caption{\label{tab:martabdb}CHILDES databases chosen for this project.}
\footnotesize
\end{margintable}

These two fundamentally different views of words mean that morphological information is stored in different ways. CoNLL-U format keeps all relevant features of a word in the token's \texttt{feats} field. And although each language in UD has its subset of morphological features, the feature types and values are by convention the same. Compared with UD, CHILDES's way of handling features is less direct. Not all features are explicit as UD's feature-value pairs. Instead, they have to be inferred from the morpheme codes or POS tags.

Listing of all possible MOR codes for grammatical morphemes is, again, unlikely. Therefore, I chose the easier path of collecting these codes from the CHILDES databases that I decided to work with. Table \ref{tab:martabdb} lists the corpora of my choice. Of all morpheme codes found in these corpora, not all of them correspond to UD features. Since morphological features are not mandatory, I only created mappings for codes with a corresponding UD-style feature-value pair. An example mapping is given in Table \ref{tab:martabfeats} on the right-hand side. The features that can be obtained from such codes are limited to 9 types: \texttt{Number}, \texttt{Person}, \texttt{Gender}, \texttt{Mood}, \texttt{Degree}, \texttt{Tense}, \texttt{Verbform}, \texttt{Voice}, and \texttt{Poss}. All possible UD features and values allowed for Engish, French, and Italian can be found in Table \ref{tab:featsmap} in \Cref{appendixa}. Notice that since the codes are sometimes capitalised and sometimes not, they are all converted to lowercase for easy comparison. The original morpheme codes are kept in the \texttt{MISC} field, which keeps all token-level information that does not fit into the previous nine fields.

\begin{margintable}[1\baselineskip]
\begin{tabularx}{1\textwidth}{@{}ll@{}}
\toprule
\textbf{MOR} & \textbf{feats}\\ \midrule
    3s & Number=Sing|Person=3\\
    cp & Degree=Cmp\\
    f & Gender=Fem\\
    inf & VerbForm=Inf\\
    imp & Mood=Imp\\
    impf & Tense=Imp\\
    pass & Voice=Pass\\
    pastp & Tense=Past|VerbForm=Part\\
    poss & Poss=Yes\\\bottomrule
\end{tabularx}
\caption{\label{tab:martabfeats}Example MOR grammatical morpheme codes and their corresponding UD featue-value pairs.}
\footnotesize
\end{margintable}


% converting GRA codes to UD deprels
\section{Dependency Relations}
\label{sec:deprel}

In CHAT annotations, syntactic relations are represented by Grammatical Relations (GRs), first devised by Kenji Sagae (\cite{sagae-etal-2004-adding}) for the English language. After later revisions in 2007, CHAT now has 37 dependency relations (\cite{sagae2007}). They correspond to \emph{deprels} (dependency relations) in CoNLL-U annotations. However, CHILDES and UD use different labels for these dependency relations and quite different parsing schemes. Sometimes, a conversion from GR to deprel also introduces a change in the structure of the dependency tree.

Defining mappings between a GR and a UD deprel thus ranges from direct label translation to linguistic knowledge-based conditional decisions. Direct mappings are listed in Table \ref{tab:martabgr}. Explanations of the adaptations made will be elaborated below and with detailed examples for a selected few hard-to-decide cases.

\begin{margintable}[1\baselineskip]
\begin{tabularx}{1\textwidth}{@{}Ll@{}}
\toprule
\textbf{GR} & \textbf{deprel}\\ \midrule
SUBJ & nsubj\\
CSUBJ & csubj\\
OBJ & obj\\
OBJ2 & iobj\\
APP & appos\\
AUX & aux\\
CONJ & conj\\
COORD & cc\\
DET & det\\
PUNCT & punct\\
ROOT & root\\\bottomrule
\end{tabularx}
\caption{\label{tab:martabgr}Example CHILDES grammatical relations and their corresponding UD dependency relations.}
\footnotesize
\end{margintable}

\paragraph{\texttt{INCROOT}}
Since CHAT transcripts authentically reflect conversational speech, the utterances may not be fully grammatical. This ungrammaticality problem becomes even more apparent when dealing with child speech. For incomplete utterances or sentences missing the main verb, a substitute root \texttt{INCROOT} is given by GRASP. However, most of the time, it does not hurt to give it the regular \texttt{ROOT} label.
\clearpage

\begin{minipage}{\widefigurewidth}
\begin{dependency}[edge slant=3pt]
    \begin{deptext}[column sep=0.7cm]
    well \& ‡ \& that \& 's \& life \& .\\
    % \textlf{1}|\textlf{0}|BEG \& \textlf{2}|\textlf{1}|BEGP \& \textlf{3}|\textlf{4}|SUBJ \& \textlf{4}|\textlf{0}|ROOT \& \textlf{5}|\textlf{4}|OBJ \& \textlf{6}|\textlf{4}|PUNCT\\
    \end{deptext}
    \depedge{1}{2}{BEGP}
    \depedge{4}{3}{SUBJ}
    \depedge{4}{5}{OBJ}
    \depedge{4}{6}{PUNCT}
    \deproot{1}{BEG}
    \deproot{4}{ROOT}
\end{dependency}
\hfill
\begin{dependency}[edge slant=3pt]
    \begin{deptext}[column sep=0.7cm]
    well \& , \& that \& 's \& life \& .\\
    \end{deptext}
    \depedge{5}{1}{discourse}
    \depedge{5}{2}{punct}
    \depedge{5}{3}{nsubj}
    \depedge{5}{4}{cop}
    \depedge{5}{6}{punct}
    \deproot[edge unit distance=4ex]{5}{root}
\end{dependency}
\end{minipage}
\captionof{figure}{CHILDES vs UD dependency graphs: interjections}\label{fig:intj}

\begin{minipage}{\widefigurewidth}
\begin{dependency}[edge slant=3pt]
    \begin{deptext}[column sep=0.7cm]
    son \& ‡ \& that \& 's \& life \& .\\
    % \textlf{1}|\textlf{0}|BEG \& \textlf{2}|\textlf{1}|BEGP \& \textlf{3}|\textlf{4}|SUBJ \& \textlf{4}|\textlf{0}|ROOT \& \textlf{5}|\textlf{4}|OBJ \& \textlf{6}|\textlf{4}|PUNCT\\
    \end{deptext}
    \depedge{1}{2}{BEGP}
    \depedge{4}{3}{SUBJ}
    \depedge{4}{5}{OBJ}
    \depedge{4}{6}{PUNCT}
    \deproot{1}{BEG}
    \deproot{4}{ROOT}
\end{dependency}
\hfill
\begin{dependency}[edge slant=3pt]
    \begin{deptext}[column sep=0.7cm]
    son \& , \& that \& 's \& life \& .\\
    \end{deptext}
    \depedge{5}{1}{vocative}
    \depedge{5}{2}{punct}
    \depedge{5}{3}{nsubj}
    \depedge{5}{4}{cop}
    \depedge{5}{6}{punct}
    \deproot[edge unit distance=4ex]{5}{root}
\end{dependency}
\end{minipage}
\captionof{figure}{CHILDES vs UD dependency graphs: vocatives}\label{fig:voc}

\section{Difficult Cases}\label{sec:grdifficult}
\subsection{The multiple-root problem}
In UD, every sentence has exactly one root, resulting in a single-root dependency tree. The dependency analysis generated by GRASP is, however, not always like that.\\

Sentences with non-final punctuations, like after an exclamation, a vocative expression or, close to the end of the utterance, before the tag question, after processed by GRASP, receives multiple roots. Although usually only one \texttt{ROOT} or \texttt{INCROOT} label is assigned, there are multiple words whose head is marked \textlf{0}. This, in effect, corrupts the single-root tree structure.\\

This problem occurs because exclamations and vocatives are not considered part of the clause. Tag markers are, however, for the chosen corpora, not so much of a problem and usually have their head marked to the functional root (\texttt{ROOT} or \texttt{INCROOT}), like in \Cref{fig:taggr}.\\

Consider the following sentences:

\pex~ Sentences with non-final punctuations\label{roots}
\a {[\sl Well\/}]$_{\text{\sc intj}}$, that's life.        \hfill {\sl interjection}\label{intj}
\a {[\sl  Son\/}]$_{\text{\sc voc}}$, that's life.         \hfill {\sl vocative}\label{voc}
\a That's life , {[\sl isn't it}]$_{\text{\sc tag}}$?    \hfill {\sl tag question}\label{tagq}
\xe


After running GRASP, sentences like (\ref{intj}) and (\ref{voc}) will be analysed like the left-hand side dependency graph of \Cref{fig:intj} and \Cref{fig:voc}, while UD-style dependeny analyses are shown on the right-hand side.\\


It is clear from the dependency graphs that the CHILDES trees are broken. Therefore, to produce a dependency tree from the pieces, one must define a change-head operation, which changes the token's head index to that of the root of the sentence.\\


After the change-head operation is applied, the current token with GR \texttt{BEG} should be connected to \texttt{ROOT}. Now the two examples in \Cref{fig:intj} and \Cref{fig:voc} share the same dependency structures, but the dependency relation labels should be different for vocatives and interjections. chatconllu needs to know the POS tags of the token with \texttt{BEG} as GR to be able to make the final decision. The decision process is outlined below:\\

\begin{itemize}
    \item If the token has a POS tag and it is not \texttt{INTJ}, \texttt{PROPN}, or \texttt{NOUN}, the corresponding \texttt{deprel} \sidenote{Definition for \texttt{parataxis} can be found from \url{https://universaldependencies.org/u/dep/parataxis.html}} label should be \texttt{parataxis} for a loose side-by-side placement between the elements.
    \item If the POS tag is \texttt{INTJ}, the corresponding \texttt{deprel} label should be \texttt{discourse} \sidenote{Definition for \texttt{discourse} can be found from \url{https://universaldependencies.org/u/dep/discourse.html}}.
    \item If the POS tag is \texttt{PROPN} or \texttt{NOUN}, the corresponding \texttt{deprel} label should be \texttt{vocative} \sidenote{Definition for \texttt{vocative} can be found from \url{https://universaldependencies.org/u/dep/vocative.html}}.
\end{itemize}

For instance, the sentence "well, son, that's just life, isn't it ?" will be analysed by GRASP as this:

% \begin{minipage}{\widefigurewidth}
\begin{minipage}{\widefigurewidth}
\begin{dependency}[edge slant=3pt]
    \begin{deptext}[column sep=1cm]
    that \& 's \& life \& „ \& is \& n't \& it \& ?\\
    % \textlf{1}|\textlf{2}|SUBJ \& \textlf{2}|\textlf{0}|ROOT \& \textlf{3}|\textlf{2}|OBJ \& \textlf{4}|\textlf{5}|ENDP \& \textlf{5}|\textlf{2}|END \& \textlf{6}|\textlf{2}|PUNCT\\
    \end{deptext}
    \depedge{2}{1}{SUBJ}
    \depedge{2}{3}{PRED}
    \depedge{5}{4}{ENDP}
    \depedge[edge unit distance=1.8ex]{2}{5}{END}
    \depedge{5}{6}{NEG}
    \depedge[edge unit distance=2.5ex]{5}{7}{PRED}
    \depedge[edge unit distance=1.5ex]{2}{8}{PUNCT}
    \deproot[edge unit distance=4ex]{2}{ROOT}
\end{dependency}
% \captionof{figure}{CHILDES}
\end{minipage}
\captionof{figure}{CHILDES: tag questions}
\label{fig:taggr}

% \begin{minipage}{\widefigurewidth}\label{fig:tagdeprel}
\begin{minipage}{\linewidth}
% \begin{dependency}[arc edge, arc angle=70, text only label, label style={above}]
\begin{dependency}[edge slant=3pt]
    \begin{deptext}[column sep=1cm]
    that \& 's \& life \& , \& is \& n't \& it \& ?\\
    \end{deptext}
    \depedge{3}{1}{nsubj}
    \depedge{3}{2}{cop}
    \depedge{3}{4}{punct}
    \depedge{3}{5}{parataxis}
    \depedge{5}{6}{advmod}
    \depedge{5}{7}{nsubj}
    \depedge[edge unit distance=1.8ex]{3}{8}{punct}
    \deproot[edge unit distance=3ex]{3}{root}
\end{dependency}
% \captionof{figure}{UD}
\end{minipage}
\captionof{figure}{UD: tag questions}
\label{fig:tagdeprel}

\clearpage
\begin{minipage}[b]{0.5\linewidth}
\begin{dependency}
    \begin{deptext}[column sep=0.5cm]
    She \& is \& a \& student \& .\\
    % PRON \& AUX \& DET \& NOUN \& PUNCT\\
    \end{deptext}
    \depedge{4}{1}{nsubj}
    \depedge{4}{2}{aux}
    \depedge{4}{3}{det}
    \depedge{4}{5}{punct}
    \deproot{4}{root}
\end{dependency}
\end{minipage}
\hfill
\begin{minipage}[b]{0.5\linewidth}
\begin{dependency}
    \begin{deptext}[column sep=0.5cm]
    She \& is \& a \& student \& .\\
    % PRON \& AUX \& DET \& NOUN \& PUNCT\\
    \end{deptext}
    \depedge{2}{1}{SUBJ}
    \depedge{2}{4}{PRED}
    \depedge{4}{3}{DET}
    \depedge{2}{5}{PUNCT}
    \deproot{2}{ROOT}
\end{dependency}
\end{minipage}

\begin{minipage}[b]{0.5\linewidth}
\begin{dependency}
    \begin{deptext}[column sep=0.5cm]
    She \& became \& happier \& .\\
    % PRON \& VERB \& ADJ \& PUNCT\\
    \end{deptext}
    \depedge{2}{1}{nsubj}
    \depedge{2}{3}{xcomp}
    \depedge{2}{5}{punct}
    \deproot{2}{root}
\end{dependency}
\end{minipage}
\begin{minipage}[b]{0.5\linewidth}
\begin{dependency}
    \begin{deptext}[column sep=0.5cm]
    She \& became \& happier \& .\\
    % PRON \& VERB \& ADJ \& PUNCT\\
    \end{deptext}
    \depedge{2}{1}{SUBJ}
    \depedge{2}{3}{PRED}
    \depedge{2}{5}{PUNCT}
    \deproot{2}{ROOT}
\end{dependency}
\end{minipage}
