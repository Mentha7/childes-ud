\chapter{Ideas} % Main chapter title

\label{Chapter7} % For referencing the chapter elsewhere, use \ref{Chapter7}

% motivation?
Unlike most written text or lab speech data, spontaneous speech is <chaotic> in form. Without drafting or editing, a sentence settle in form as each word is spoken. Errors and markers of disfluency like pauses and repetitions are ubiquitous in spontaneous speech data. Before a consensus is reached in how people transcribe and annotate speech, transcribers had to device their own markings/symbols to faithfully transcribe the utterances/conversational interactions. The disadvantages are clear, inconsistent use of symbols is an obstacle for the sharing of linguistic data, which is indispensable in Natural Language Processing.\\

% references
CHILDES database \cite{Macwhinney2000}\\
the Brown Corpus <needs bibtex (Brown, 1973)>\\
