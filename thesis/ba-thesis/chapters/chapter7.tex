\chapter{Ideas} % Main chapter title

\label{Chapter7} % For referencing the chapter elsewhere, use \ref{Chapter7}

% motivation?
Unlike most written text or lab speech data, spontaneous speech is <chaotic> in form. Without drafting or editing, a sentence settle in form as each word is spoken. Errors and markers of disfluency like pauses and repetitions are ubiquitous in spontaneous speech data. Before a consensus is reached in how people transcribe and annotate speech, transcribers had to device their own markings/symbols to faithfully transcribe the utterances/conversational interactions. The disadvantages are clear, inconsistent use of symbols is an obstacle for the sharing of linguistic data, which is indispensable in Natural Language Processing.\\

% references
CHILDES database \cite{Macwhinney2000}\\
the Brown Corpus <needs bibtex (Brown, 1973)>\\

\section{Mappings Related} % (fold)
\label{sec:mappings}

% section section_name (end)
% converting MOR codes to UD Universal POS tags
\subsection{POS tags mappings}

\paragraph{\texttt{art}}
Articles are marked as \texttt{DET} in UD Treebanks with feature \texttt{PronType=Art}.\\
\paragraph{\texttt{preart}}
\texttt{preart} are contractions of prepositions and articles, for instance the Italian \emph{alla} is the contraction of \emph{a} and \emph{la}. In UD-style annotations, words like \emph{alla} are treated as multi-word tokens and a POS tag is given to each part of the token. However, CHAT consider them to be single words with forms of their own. To keep the alignment with \%mor tier, here \texttt{preart} is mapped to \texttt{DET} because the word functions as a determiner.\\ 

\paragraph{\texttt{vimp}}
Imperatives are tagged with \texttt{vimp} in \%mor tier. However, in UD, imperatives are tagged with \texttt{VERB}, which is the category for all finite verbforms. Information of gender, person, number, tense, aspect, mood and voice are all specified as features. Therefore \texttt{vimp}, as an example, is mapped to \texttt{VERB} with \texttt{MOOD=Imp}.\\

\paragraph{\texttt{vpfx}}
Preverb/verb particles are tagged with \texttt{vpfx}. Although they are particles, according to UD's definition <insert reference to UD website in sidenote>, they belong to \texttt{ADP}.\\

\paragraph{Tags with no UPOS counterparts}


\subsection{GR-deprel mappings}

\paragraph{\texttt{INCROOT}}

\paragraph{\texttt{PRED}}



