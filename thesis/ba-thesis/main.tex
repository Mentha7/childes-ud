%%%%%%%%%%%%%%%%%%%%%%%%%%%%%%%%%%%%%%%%%
%
% PSI Chair Thesis Template
% Version 20200913
%
% based on MastersDoctoralThesis.cls
% Version 2.5 (27/8/17)
%
% which was obtained from:
% http://www.LaTeXTemplates.com
%
% Version 2.x major modifications by:
% Vel (vel@latextemplates.com)
%
% This template is based on a template by:
% Steve Gunn (http://users.ecs.soton.ac.uk/srg/softwaretools/document/templates/)
% Sunil Patel (http://www.sunilpatel.co.uk/thesis-template/)
%
% License of this guide and the template
% CC BY-SA 4.0 (http://creativecommons.org/licenses/by-sa/4.0/)
%
% Exception 1: Some excerpts, figures, and tables that have been taken
% from the literature (denoted with a citation in the caption) are not
% covered by the above license. Permission to re-use and distribute
% these excerpts, figures, and tables must be obtained from the
% respective holder of the copyrights.
%
% Exception 2: Chapter 1 and Appendix C are based on content from the
% MastersDoctoralThesis template mentioned above, which is licensed under
% CC BY-SA 3.0 (http://creativecommons.org/licenses/by-nc-sa/3.0/)
%
% License of the PSIThesis.cls class file:
% LPPL v1.3c (http://www.latex-project.org/lppl)
%
%%%%%%%%%%%%%%%%%%%%%%%%%%%%%%%%%%%%%%%%%

%----------------------------------------------------------------------------------------
% PACKAGES AND OTHER DOCUMENT CONFIGURATIONS
%----------------------------------------------------------------------------------------

\PassOptionsToPackage{english,ngerman}{babel}
\documentclass[
11pt, % The default document font size is 11 (recommended), options: 10pt, 11pt, 12pt
oneside, % Two-side layout is recommended; uncomment to switch to one-sided
english, % replace with ngerman for German; not fully supported so far -- requires changes elsewhere
singlespacing, % Single line spacing (recommended), alternatives: onehalfspacing or doublespacing
% draft, % Uncomment to enable draft mode (no pictures, no links, overfull hboxes indicated)
%nolistspacing, % If the document is onehalfspacing or doublespacing, uncomment this to set spacing in lists to single
%liststotoc, % Uncomment to add list of figures/tables/etc to table of contents (not recommended)
%toctotoc, % Uncomment to add the main table of contents to the table of contents (not recommended)
parskip, % add space between paragraphs (recommended)
nohyperref, % do not load the hyperref package (is loaded in setup.tex)
%headsepline, % print a horizontal line under the page header
consistentlayout, % layout of declaration, abstract and acknowledgements pages matches the default layout
%final, % Uncomment to hide all todo notes
]{PSIThesis} % The class file specifying the document structure

% version of the guide
\def\tversion{v20200913}

% long-term stable URL to the thesis guide
\def\doiurl{https://doi.org/10.20378/irb-48428}
\def\githuburl{https://github.com/UBA-PSI/psi-thesis-guide}

\input{misc/setup.tex} % Load the settings from Misc/setup.tex
\input{misc/commands.tex} % Load the custom commands from Misc/commands.tex

% Uncomment this command to make all links black:
%   useful for printing on black-white printers that do a
%   poor job at rasterizing colored text properly
%\hypersetup{colorlinks=false}

\addbibresource{literature.bib} % The filename of the bibliography

%----------------------------------------------------------------------------------------
% THESIS INFORMATION
%----------------------------------------------------------------------------------------
\newcommand{\thesistype}{Bachelor} % type of your thesis (Bachelors, Masters, Doctoral ...)

%%% CHANGE THIS:
% Your thesis title, this is used in the title and abstract, print it elsewhere with \ttitle
\thesistitle{chatconllu: a Two-way Converter between CHILDES and UD Annotations}

% date to be printed on the title, this will automatically update and be in the correct format
% If any changes to this format (Month JJJJ) are necessary the definition can be found in line 337
% of misc/setup.tex
\def\tdate{\monthyeardate\today}

%%% CHANGE THIS:
% Your name, this is used in the title page, print it elsewhere with \authorname
\author{Jingwen Li}

% Your supervisor's name, this is used in the title page, print it elsewhere with \supname
\supervisor{Dr. \c{C}a\u{g}r{\i} \c{C}\"{o}ltekin}

% Your university's name and URL, this is used in the title page, print it elsewhere with \univname
\university{\href{https://uni-tuebingen.de/en/}{Eberhard Karls\\ Universit\"{a}t T\"{u}bingen}}

% Your research group's name and URL, this is used in the title page, print it elsewhere with \groupname
\group{\href{https://uni-tuebingen.de/fakultaeten/philosophische-fakultaet/fachbereiche/neuphilologie/seminar-fuer-sprachwissenschaft/}{Seminar f\"{u}r Sprachwissenschaft}}

% Your department's name and URL, this is used in the title page, print it elsewhere with \deptname
\department{Philosophische Fakult\"{a}t}

% Your faculty's name and URL, this is used in the title page, print it elsewhere with \facname
% TODO: insert *your* degree program in the \faculty command below
% Applied Computer Science
% Computing in the Humanities
% Information Systems
% International Information Systems Management
% International Software Systems Science
% Software Systems Science
% Education in Business and Information Systems
\faculty{Seminar f\"{u}r Sprachwissenschaft\\ \href{https://uni-tuebingen.de/fakultaeten/philosophische-fakultaet/fachbereiche/neuphilologie/seminar-fuer-sprachwissenschaft/}{Philosophische Fakult\"{a}t}}

% Your address, this is not currently used anywhere in the template, print it elsewhere with \addressname
\addresses{address not used}

% Your subject area, this is not currently used anywhere in the template, print it elsewhere with \subjectname
\subject{subject not used}

% Keywords for your thesis, this is not currently used anywhere in the template, print it elsewhere with \keywordnames
\keywords{keywords not used}
%----------------------------------------------------------------------------------------
% END OF THESIS INFORMATION
%----------------------------------------------------------------------------------------


\begin{document}
\hyphenation{CoNLL-U CHAT CLAN MOR GRASP POST}

\selectlanguage{english}

\frenchspacing % do not add additional hspace after end of sentence full stop dot.

\frontmatter % Uses roman page numbering style (i, ii, iii, iv...) for the pre-content pages

\hypersetup{urlcolor=black}

\include{misc/titlepage} % Typeset the titlepage

\hypersetup{urlcolor=triadicblue}


%----------------------------------------------------------------------------------------
% QUOTATION
%----------------------------------------------------------------------------------------

% \vspace*{0.2\textheight}

% \noindent\enquote{\itshape Thanks to my solid academic training, today I can write hundreds of words on virtually any topic without possessing a shred of information, which is how I got a good job in journalism.}\bigbreak

% \hfill Dave Barry


%----------------------------------------------------------------------------------------
% ABSTRACT PAGE
%----------------------------------------------------------------------------------------

\begin{abstract}
%\addchaptertocentry{\abstractname}
% uncomment to add the abstract to the table of contents (not recommended)

Both \emph{Child Language Data Exchange System} (CHILDES) and \emph{universal dependencies} (UD) are projects designed to facilitate linguistic research with computerised tools. Each project is paired with a standardised annotation format, which removes barriers to linguistic data-sharing introduced by diverse annotation or transcription habits.

The CHILDES \emph{CHAT} and UD \emph{CoNLL-U} formats, although widely accepted and used in their dominant fields, use different structures and tagsets to represent morphosyntax. With the help of a two-way converter, information can be augmented or cross-validated using existing tools developed for these two annotation formalisms.

To improve the interoperability between CoNLL-U and CHAT, a deterministic mapping for morphosyntactic information from CHAT codes to UD-style tags is created, and \emph{chatconllu} is implemented as a command-line-based two-way converter. Within the limits of a Bachelor thesis, \emph{chatconllu} is developed for and tested on several selected CHILDES databases of English, French and Italian, and the converted files are validated using CHILDES- and UD-maintained tools.

\end{abstract}


%----------------------------------------------------------------------------------------
% ACKNOWLEDGEMENTS
%----------------------------------------------------------------------------------------

%\begin{acknowledgements}
% %\addchaptertocentry{\acknowledgementname}
% Add the acknowledgements to the table of contents (not recommended)
%
%The acknowledgments and the people to thank go here.
%\end{acknowledgements}


%----------------------------------------------------------------------------------------
% TABLE OF CONTENTS
%----------------------------------------------------------------------------------------

\cleardoublepage

% Table of Contents uses a wider layout than the main content
\newgeometry{
        head=13.6pt,
        top=27.4mm,
        bottom=27.4mm,
        inner=24.8mm,
        outer=24.8mm,
        marginparsep=0mm,
        marginparwidth=0mm,
}
{
\hypersetup{linkcolor=black}
\tableofcontents % Prints the ToC entries
}
\restoregeometry

%----------------------------------------------------------------------------------------
% DEDICATION
%----------------------------------------------------------------------------------------

% \dedicatory{For/Dedicated to/To my\ldots}


%----------------------------------------------------------------------------------------
% THESIS CONTENT - CHAPTERS
%----------------------------------------------------------------------------------------
\mainmatter % From here on, numeric (1,2,3...) page numbering
\pagestyle{thesis} % Return the page headers back to the "thesis" style

% Define some commands to keep the formatting separated from the content
\newcommand{\keyword}[1]{\textbf{#1}}
\newcommand{\tabhead}[1]{\textbf{#1}}
\newcommand{\code}[1]{\texttt{#1}}
\newcommand{\file}[1]{\texttt{#1}}
\newcommand{\option}[1]{\texttt{\itshape#1}}

% Figures will automatically be searched for in the Figures subdirectory
\graphicspath{{./figures/}{./examples/}}

%%% CHANGES NEEDED HERE
%
% Include the chapters of the thesis as separate files from the Chapters folder
% Uncomment the lines as you write the chapters
% Mind the \input instead of the \include here, that change is necessary for the appendix formatting
% Due to the \input command you also need to provide the .tex file ending

\chapter{Introduction} % Main chapter title

\label{Chapter1} % For referencing the chapter elsewhere, use \ref{Chapter1}

% explain the choice of the 2 annotations first

The focus of this thesis is to explore and experiment with format conversion between these two annotation schemes through designing and implementing a two-way converter with Python. The scope of analysis is limited to a selected few corpora from the CHILDES database. Transcriptions in \emph{CHAT} are first converted to \emph{CoNLL-U} and then converted back to see which information are lost in the process.


% explain the choice of the 2 annotations, to be finished separately and integrated into the section above.
\section{CHILDES}

The Child Language Data Exchange System (CHILDES) is part of TalkBank project with focus on child language.

\section{UD}


\chapter{CHILDES \emph{CHAT} and UD \emph{CoNLL-U}}
\label{Chapter2}

This chapter introduces the two annotation schemes of interest---\emph{CHAT} and \emph{CoNLL-U}---in detail and gives the reader an idea of what information is shared by both annotations and what information is exclusively relevant to only one of them. This will shed light on the implementation of \emph{chatconllu} which will be discussed in \Cref{Chapter3}.

\section{CHILDES CHAT Transcription System}

\subsection{Motivation}
The \emph{Codes for Human Analysis of Transcripts} (CHAT) transcription system is developed for the (\cite{Macwhinney2000})\emph{Child Language Data Exchange System} (CHILDES) project (\cite{Macwhinney2000}) to study child language acquisition.

As MacWhinney pointed out in the CHAT manual, collecting audio data for spontaneous conversational interactions is easy, one just have to turn on the recorder, however, processing the collected recordings and turning them into usable data for linguistic research is a demanding task (\cite{Macwhinney2000}). Transcribing the conversations is such an indispensable preprocessing step. Consistency treatment of raw data is critical to scientific research. Therefore, it is hard to imagine how difficult it was to do consistent transcription when there was no standard format, especially for child speech, which is usually more chaotic than average adult speech data. Thus CHAT is designed to accommodate the need for a standard transcription system of conversational speech. Moreover, CHAT transcripts can be analysed by TalkBank programs \sidenote{\url{https://talkbank.org/software/}} like \emph{Computerized Language Analysis} (commonly referred to as the CLAN Program), which facilitates the transcription, sharing and analysis of human conversational interactions.

\subsection{File Structure}
For a CHAT file to be well-formed, there are some basic structural requirements to comply with. The minimum standards are prescribed by \emph{minCHAT} (\cite{Macwhinney2000}).

% describe only relevant minCHAT structure
First of all, each CHAT file should have 2 major parts:\\
\begin{itemize}
	\item headers that stores metadata like language and participants
	\item utterances which are the actual sentences being said
\end{itemize}
Each sentence is represented by a single line beginning with an asterisk symbol \texttt{*}, this is also called the \emph{main line}. Additional information can be optionally added as dependent tiers. Annotations of morphological, syntactical, and phonological information or general remarks of the transcriber's
choice all go into separate dependent tiers of the main utterance. Dependent tiers are preceded by the \texttt{\%} symbol.
% describe mor and gra tiers
\subsubsection{Dependent Tiers}
Almost any type of information can be organised into separate dependent tiers and added to each utterance. The two most commonly seen tiers are \texttt{\%mor} and \texttt{\%gra} which can be generated automatically by the use of a series of CLAN programs---\emph{MOR}, \emph{POST}, \emph{POSTMORTEM} and \emph{MEGRASP}.

Since information in \texttt{\%mor} and \texttt{\%gra} should be kept with individual tokens in CoNLL-U files, they have to be processed in such a way that all relevant information from both tiers (if they are present) are associated with the respective token in the utterance.

\subsubsection{Morphosyntactic Coding with \emph{\%mor} Tier}
% token links %mor to the main line. one-to-one correspondence between
normalised utterance and %mor.
% structure of individual words
% structure of word groups  -> conllu: multi-word tokens

\subsubsection{Syntactic Dependency Analysis with \emph{\%gra} Tier}

\section{UD \emph{CoNLL-U} Annotation Scheme}

\section{Shared Information vs. Exclusive Information}
% token as the basic unit of CHAT and CoNLL-U annotations.
% token also links %mor and %gra to the main line. one-to-one correspondence
% between normalised utterance and %mor.

\chapter{Design} % Main chapter title

\label{Chapter3} % For referencing the chapter elsewhere, use \ref{Chapter2}

To make the tool manageable within the frame of a bachelor's thesis, chatconllu is designed based on several selected corpora from the CHILDES database.

\section{General Structure of the Program}
\section{\emph{CHAT} to \emph{CoNLL-U}}
\section{\emph{CoNLL-U} to \emph{CHAT}}
\section{Organization of Outputs}

\chapter{Deterministic Mapping for Morphosyntax} % Main chapter title
\label{Chapter4}

The previous chapter talked about how CHAT and CoNLL-U use \texttt{MOR} and \texttt{GRA} codes to structurally represent morphosyntactic information. However, the set of tags used in the two annotation schemes could also be semantically different. This chapter presents a deterministic mapping from CHAT codes to UD-style labels. The mapping consists of three major categories: part-of-speech tags, morphological features and dependency relations and is created for English, French and Italian.

% converting MOR codes to UD Universal POS tags
\section{Part-of-Speech Tags}
\label{sec:pos}
% left-aligned table columns with automated line breaks
\newcolumntype{L}{>{\RaggedRight\arraybackslash}X}
\begin{margintable}[1\baselineskip] % move figure down by 1 line
\begin{tabularx}{\textwidth}{@{}Ll@{}}
\toprule
\textbf{MOR} & \textbf{UD}\\ \midrule
adj & ADJ\\
adv & ADV\\
coord & CCONJ\\
n & NOUN\\
num & NUM\\
pro & PRON\\
n:prop & PROPN\\
comp & SCONJ\\
v & VERB\\
det, qn & DET\\
co, fil & INTJ\\
dia, none & X\\
post, prep & ADP\\
aux, cop, mod & AUX\\
beg, end & PUNCT\\
inf, neg, part & PART\\\bottomrule
\end{tabularx}
\caption{\label{tab:martabpos}Examples of MOR POS categories that can be directly converted to UPOS tags. See Table \ref{tab:posmap1} and Table \ref{tab:posmap2} in \Cref{appendixa} for the full mapping.}
\footnotesize
% Examples of MOR POS categories that can be directly converted to UPOS tags. See \ref{tab:posmap1} and \ref{tab:posmap2} in \Cref{appendixa} for the full mapping.
\end{margintable}


The CHAT manual listed 39 POS tags for English (\cite{Macwhinney2000}, Part 3, p.8). This set can be extended with language-specific tags for other languages. UD, on the other hand, uses a universal POS (UPOS) tagset with only 17 tags for all languages in the UD project.

UD's UPOS is developed based on Petrov's \emph{A Universal Part-of-Speech Tagset}, where the crosslinguistic universality of the UPOS tags stems from their operational definitions instead of presupposing the existence of actual grammatical categories (\cite{petrov2012}). In CHAT, however, POS tags are much less universal across languages. Typologically different languages may use quite different tagsets than that prescribed by the MOR program for English (\cite{Macwhinney2000}, Part3, p.7). Additionally, researchers can define their own POS tagsets, which not only means that, in terms of POS tags, CHILDES took an opposite approach from UD, but also makes an exhaustive mapping between POS tags used in the two annotation schemes impossible. Therefore, the mapping I define here concerns only the standard POS tagset given by the MOR grammar for English, French and Italian.

The adaptation from MOR POS categories to UD's UPOS can be achieved in the following ways:
\begin{itemize}
	\item create a direct mapping between two tags if they share the same semantics or have a straightforward relationship
	\item collapse MOR POS categories into one UPOS tag when applicable
	\item make best-effort decisions based on the description of the POS cetegories by both formalisms
	\item map MOR POS categories with no UPOS counterparts to \texttt{X}
\end{itemize}

An example of a straightforward mapping between MOR POS codes and UPOS tags is shown in Table \ref{tab:martabpos}. A total of 16 out of the 17 UPOS tags are used in the tagset mapping, with the exception of \texttt{SYM}, which is reserved for symbols.

Collapsing of POS categories can be illustrated by the example of auxiliary verbs. The MOR code differentiates between auxiliary verbs \texttt{aux} or \texttt{v:aux}, copulas \texttt{cop} and modal verbs \texttt{mod}. However, since UD took an operational approach to define grammatical categories, it does not distinguish copulas or modal verbs from other verbs that function as auxiliaries. By doing so, UPOS can be applied to typologically distant languages.

The following MOR POS codes require more thoughts during the creation of a deterministic mapping, partially because they do not exist in all languages. For tags like these, UD usually raises them to a more coarse category and represents the exact type using morphological features instead.

\paragraph{\texttt{art}}
Articles are ganerally marked as \texttt{DET} in UD Treebanks with feature \texttt{PronType=Art} \sidenote{Definition for \texttt{DET} can be found from \url{https://universaldependencies.org/docs/u/pos/DET.html}}.

\paragraph{\texttt{preart}}
\texttt{preart} are contractions of prepositions and articles, for instance the Italian \emph{alla} is the contraction of \emph{a} and \emph{la}. In UD-style annotations, words like \emph{alla} are treated as multi-word tokens, and a POS tag is given to each part of the token. However, CHAT considers them to be single words with forms of their own. To keep the alignment with the \texttt{\%mor} tier, here \texttt{preart} is mapped to \texttt{DET} because the word functions as a determiner.

\paragraph{\texttt{vimp}}
Imperatives are tagged with \texttt{vimp} in \%mor tier. However, in UD, imperatives are tagged with \texttt{VERB}, which is the category for all finite verb forms \sidenote{Definition for \texttt{VERB} can be found from \url{https://universaldependencies.org/docs/u/pos/VERB.html}} . In UD, gender, person, number, tense, aspect, mood and voice are all specified as features. Therefore \texttt{vimp}, listed here as an example for finite verbs, is mapped to \texttt{VERB} with feature \texttt{MOOD=Imp}.

\paragraph{\texttt{vpfx}}
Preverb/verb particles are tagged with \texttt{vpfx}. Although they are particles, according to UD's definition \sidenote{Definition for \texttt{ADP} can be found from \url{https://universaldependencies.org/docs/u/pos/ADP.html}}, they belong to \texttt{ADP}.

\paragraph{\texttt{conj}}
Despite the similarity between the MOR \texttt{conj} and UD \texttt{CCONJ}, they represent different things. UD distinguishes between \texttt{CCONJ} \sidenote{Definition for \texttt{CCONJ} can be found from \url{https://universaldependencies.org/docs/u/pos/CCONJ.html}} and \texttt{SCONJ} \sidenote{Definition for \texttt{SCONJ} can be found from \url{https://universaldependencies.org/docs/u/pos/SCONJ.html}}, the former is for coordinating conjunctions like \emph{and} and \emph{but}, which are tagged \texttt{coord} by MOR, and the latter marks subordinating conjunctions with complementizers like \emph{that} or \emph{if}, or adverbial clause introducers like \emph{when}, these are tagged \texttt{conj} by MOR. Therefore, the suitable UPOS tag for the MOR POS tag \texttt{conj} is \texttt{SCONJ}.

\paragraph{special form markers}
It is worth noticing that some MOR codes do not correspond to any grammatical categories. For example, in CHAT, the so-called \emph{special form markers} have unique POS tags independent of the part-of-speech category to which the word belongs (\cite{Macwhinney2000}, Part 1, pp. 45-47). Nevertheless, some of these tags can still be mapped to UPOS.The tags \texttt{sing} for singing, \texttt{bab} for babbling and \texttt{wplay} for wordplay can all be mapped to \texttt{INTJ}, which stands for \emph{interjection}. The definition of interjection, however, is vague (\cite{ameka1992}). UD defines interjection descriptively \sidenote{"An interjection is a word that is used most often as an exclamation or part of an exclamation. It typically expresses an emotional reaction, is not syntactically related to other accompanying expressions, and may include a combination of sounds not otherwise found in the language." Excerpt from \url{https://universaldependencies.org/docs/u/pos/SCONJ.html}}. Following UD's definition for \texttt{INTJ}, the core criteria for a word \sidenote{or non-word in this case} to be an interjection include:
\begin{itemize}
	\item that it is often used exclamatorily
	\item that it typically expresses emotions
	\item that it plays no syntactic role in the utterance
	\item that it might not be a recognised word in the language
\end{itemize}
Since singing, babbling and wordplay satisfy these criteria, along with the MOR code \texttt{co} for interjections , they are granted the \texttt{INTJ} tag. Other special form markers, on the other hand, have no corresponding UPOS tags whatsoever, for example \texttt{dia} for dialect forms. Unfortunately they have to be assigned to \texttt{X}, as words that does not belong to the other 16 categories.

\begin{margintable}[1\baselineskip] % move figure down by 1 line
\begin{tabularx}{\textwidth}{@{}cl@{}}
\toprule
\textbf{Punctuations} & \textbf{MOR code}\\\midrule
‡ & beg\\
, & cm\\
„ & end\\
“ & bq\\
” & eq\\
‘ & bq2\\
’ & eq2\\\bottomrule
\end{tabularx}
\caption{\label{tab:martabpunct}MOR punctuation marks and their corresponding MOR codes.}
\footnotesize
\end{margintable}

\paragraph{punctuation marks}
Finally, MOR also defines a set of punctuation marks with string representations, shown in Table \ref{tab:martabpunct}. Although treated as lexical items in the \%mor tier, they are all given the \texttt{PUNCT} tag.

% feature mappings
\section{Morphological Features}
\label{sec:feats}

Since \texttt{MOR} uses a morpheme-based approach to analyse individual words, morphological features like \emph{tense} and \emph{mood} are encoded by morpheme codes. Unlike MOR's dissecting analysis of a word, UD takes a lexicalist view on morphology and treats word as a whole lexical item. Features are thus considered properties of the word. The existence of morphemes is not presupposed.

\begin{margintable}[1\baselineskip]
\begin{tabularx}{1\textwidth}{@{}ll@{}}
\toprule
\textbf{CHILDES database} & \textbf{Language}\\ \midrule
	Belfast & eng\\
	Brown & eng\\
	English-MiamiBiling & eng\\
	Wells & eng\\
	Geneva & fra\\
	Leveille & fra\\
	MTLN & fra\\
	vioncolas & fra\\
	Tonelli & ita\\\bottomrule
\end{tabularx}
\caption{\label{tab:martabdb}CHILDES databases chosen for this project.}
\footnotesize
\end{margintable}

These two fundamentally different views of words are reflected in the different ways of structuring morphological information. CoNLL-U format keeps all relevant features of a word in the token's \texttt{FEATS} field. And although each language in UD has its subset of morphological features, the feature types and values are by convention the same. Compared with UD, CHILDES's way of handling features is less direct. Not all features are explicit as UD's feature-value pairs. Instead, they have to be inferred from the morpheme codes or POS tags.

Listing of all possible MOR codes for grammatical morphemes is, again, unlikely. Therefore, I chose the easier path of collecting these codes from the CHILDES databases that I decided to work with, listed in \Cref{tab:martabdb}.
\begin{margintable}[1\baselineskip]
\begin{tabularx}{1\textwidth}{@{}ll@{}}
\toprule
\textbf{MOR} & \textbf{UD feats}\\ \midrule
	3s & Number=Sing|Person=3\\
	cp & Degree=Cmp\\
	f & Gender=Fem\\
	inf & VerbForm=Inf\\
	imp & Mood=Imp\\
	impf & Tense=Imp\\
	pass & Voice=Pass\\
	pastp & Tense=Past|VerbForm=Part\\
	poss & Poss=Yes\\\bottomrule
\end{tabularx}
\caption{\label{tab:martabfeats}Example MOR grammatical morpheme codes and their corresponding UD featue-value pairs.}
\footnotesize
\end{margintable}
Of all the morpheme codes found in these corpora, not all of them correspond to UD features. Since morphological features are not mandatory, I only created mappings for those with a corresponding UD feature-value pair. An example mapping is given in Table \ref{tab:martabfeats} on the right-hand side. Only 9 types of features can be obtained directly from the morpheme codes I encountered: \texttt{Number}, \texttt{Person}, \texttt{Gender}, \texttt{Mood}, \texttt{Degree}, \texttt{Tense}, \texttt{Verbform}, \texttt{Voice}, and \texttt{Poss}. All possible UD features and values allowed for Engish, French, and Italian can be found in Table \ref{tab:featsmap} in \Cref{appendixa}. The codes are sometimes capitalised and sometimes not due to language- or corpus-specific conventions. They are all converted to lowercase for easy comparison. The original morpheme codes are kept in the \texttt{MISC} field.
% \begin{margintable}[1\baselineskip]
% \begin{tabularx}{1\textwidth}{@{}ll@{}}
% \toprule
% \textbf{MOR code} & \textbf{UD feats}\\ \midrule
%     3s & Number=Sing|Person=3\\
%     cp & Degree=Cmp\\
%     f & Gender=Fem\\
%     inf & VerbForm=Inf\\
%     imp & Mood=Imp\\
%     impf & Tense=Imp\\
%     pass & Voice=Pass\\
%     pastp & Tense=Past|VerbForm=Part\\
%     poss & Poss=Yes\\\bottomrule
% \end{tabularx}
% \caption{\label{tab:martabfeats}Example MOR grammatical morpheme codes and their corresponding UD featue-value pairs.}
% \footnotesize
% \end{margintable}


% converting GRA codes to UD deprels
\section{Dependency Relations}
\label{sec:deprel}

In CHAT annotations, syntactic relations are represented by Grammatical Relations (GRs). They were first devised by Kenji Sagae (\cite{sagae-etal-2004-adding}) for the English language. After later revisions in 2007, CHAT now has 37 grammatical relation types (\cite{sagae2007}). They correspond to dependency relations stored in the \texttt{DEPRELS} field in CoNLL-U files. However, CHILDES and UD use different labels for these dependency relations and quite different parsing schemes. Sometimes, a conversion from GR to deprel also introduces a change in the structure of the dependency tree.

\begin{margintable}[1\baselineskip]
\begin{tabularx}{1\textwidth}{@{}Ll@{}}
\toprule
\textbf{GRA} & \textbf{UD deprel}\\ \midrule
SUBJ & nsubj\\
CSUBJ & csubj\\
OBJ & obj\\
OBJ2 & iobj\\
DET & det\\
APP & appos\\
AUX & aux\\
CONJ & conj\\
COORD & cc\\
PUNCT & punct\\
ROOT & root\\\bottomrule
\end{tabularx}
\caption{\label{tab:martabgr}Example CHILDES grammatical relations and their corresponding UD dependency relations.}
\footnotesize
\end{margintable}

Defining a deterministic mapping between GRs and UD dependency relations thus ranges from direct label translation to linguistically-informed decisions. \Cref{tab:martabgr} gives examples of direct label conversions. The full mapping for the current version of chatconllu can be found in \Cref{tab:grmap}.

 A solid mapping requires both familiarity with the grammatical relations defined in the two formalisms and profound linguistic knowledge for the languages involved, this mapping I created is only a first attempt with many inprecisions. Therefore, instead of explaining the decision process with my limited linguistic knowledge, I choose to identify and discuss several hard-to-decide cases and cases where CHILDES and UD provide different dependency parses. Since only the English databases are annotated with syntactic relations, all labels and discussions here apply only to the English language.

The current version of chatconllu does not handle most of the cases discussed below, because it does not seek to produce gold-standard UD-style dependency parses, but only convert the files to CoNLL-U format, which can then be processed by other dependency parsers, for example those trained on UD treebanks. Therefore, my best effort in designing this mapping goes to transforming existing GR labels to UD labels without changing the dependency structures produced by GRASP in most cases to aid comparison between the two parsing schemes.


% \subsection{Direct Label Translation}

% \paragraph{\texttt{INCROOT}}
% Since CHAT transcripts tries to faithfully transcribe conversational speech, the utterances in CHILDES corpora may not be fully grammatical. This ungrammaticality problem becomes even more apparent when dealing with child speech. For incomplete utterances or sentences missing the main verb, a substitute root \texttt{INCROOT} is given by GRASP. Most of the time, it does not hurt to give it the regular \texttt{ROOT} label directly.

\begin{minipage}{\widefigurewidth}
\begin{dependency}[edge slant=3pt]
	\begin{deptext}[column sep=0.7cm]
	well \& ‡ \& that \& 's \& life \& .\\
	% \textlf{1}|\textlf{0}|BEG \& \textlf{2}|\textlf{1}|BEGP \& \textlf{3}|\textlf{4}|SUBJ \& \textlf{4}|\textlf{0}|ROOT \& \textlf{5}|\textlf{4}|OBJ \& \textlf{6}|\textlf{4}|PUNCT\\
	\end{deptext}
	\depedge{1}{2}{BEGP}
	\depedge{4}{3}{SUBJ}
	\depedge{4}{5}{OBJ}
	\depedge{4}{6}{PUNCT}
	\deproot{1}{BEG}
	\deproot{4}{ROOT}
\end{dependency}
\hfill
\begin{dependency}[edge slant=3pt]
	\begin{deptext}[column sep=0.7cm]
	well \& , \& that \& 's \& life \& .\\
	\end{deptext}
	\depedge{5}{1}{discourse}
	\depedge{5}{2}{punct}
	\depedge{5}{3}{nsubj}
	\depedge{5}{4}{cop}
	\depedge{5}{6}{punct}
	\deproot[edge unit distance=4ex]{5}{root}
\end{dependency}
\end{minipage}
\captionof{figure}{CHILDES vs UD dependency graphs: interjections}\label{fig:intj}

\begin{minipage}{\widefigurewidth}
\begin{dependency}[edge slant=3pt]
	\begin{deptext}[column sep=0.7cm]
	son \& ‡ \& that \& 's \& life \& .\\
	% \textlf{1}|\textlf{0}|BEG \& \textlf{2}|\textlf{1}|BEGP \& \textlf{3}|\textlf{4}|SUBJ \& \textlf{4}|\textlf{0}|ROOT \& \textlf{5}|\textlf{4}|OBJ \& \textlf{6}|\textlf{4}|PUNCT\\
	\end{deptext}
	\depedge{1}{2}{BEGP}
	\depedge{4}{3}{SUBJ}
	\depedge{4}{5}{OBJ}
	\depedge{4}{6}{PUNCT}
	\deproot{1}{BEG}
	\deproot{4}{ROOT}
\end{dependency}
\hfill
\begin{dependency}[edge slant=3pt]
	\begin{deptext}[column sep=0.7cm]
	son \& , \& that \& 's \& life \& .\\
	\end{deptext}
	\depedge{5}{1}{vocative}
	\depedge{5}{2}{punct}
	\depedge{5}{3}{nsubj}
	\depedge{5}{4}{cop}
	\depedge{5}{6}{punct}
	\deproot[edge unit distance=4ex]{5}{root}
\end{dependency}
\end{minipage}
\captionof{figure}{CHILDES vs UD dependency graphs: vocatives}\label{fig:voc}

% \subsection{Difficult Cases}\label{sec:grdifficult}
\subsection{The Multiple-Root Problem}
In UD, every sentence has exactly one root, represented by a single-root dependency tree. The dependency analysis generated by GRASP is, however, not always like that. After processed by GRASP, sentences with non-final punctuations, like after an exclamation, a vocative expression or, close to the end of the utterance, before the tag question, receives multiple roots. Although usually only one \texttt{ROOT} or \texttt{INCROOT} label is assigned, there can be multiple words whose head is marked \textlf{0}, which, in effect, corrupts the single-root tree structure.

This problem occurs because exclamations and vocatives are not considered part of the clause. Tag markers are, however, for the chosen few corpora, not so much of a problem and usually have their head marked to the functional root \sidenote{Although from my observations, tag questions present a different problem of inconsistent parses of the same structure.}(\texttt{ROOT} or \texttt{INCROOT}), like in \Cref{fig:taggr}.

Consider the following sentences:

\pex~ Sentences with non-final punctuations\label{roots}
\a {[\sl Well\/}]$_{\text{\sc intj}}$, that's life.        \hfill {\sl interjection}\label{intj}
\a {[\sl  Son\/}]$_{\text{\sc voc}}$, that's life.         \hfill {\sl vocative}\label{voc}
\a That's life , {[\sl isn't it}]$_{\text{\sc tag}}$?    \hfill {\sl tag question}\label{tagq}
\xe


After running GRASP, sentences like (\ref{intj}) and (\ref{voc}) will be analysed like the left-hand side dependency graph of \Cref{fig:intj} and \Cref{fig:voc}, while UD-style dependeny analyses are shown on the right-hand side.\\

\begin{minipage}{\widefigurewidth}
\begin{dependency}[edge slant=3pt]
	\begin{deptext}[column sep=1cm]
	that \& 's \& life \& „ \& is \& n't \& it \& ?\\
	% \textlf{1}|\textlf{2}|SUBJ \& \textlf{2}|\textlf{0}|ROOT \& \textlf{3}|\textlf{2}|OBJ \& \textlf{4}|\textlf{5}|ENDP \& \textlf{5}|\textlf{2}|END \& \textlf{6}|\textlf{2}|PUNCT\\
	\end{deptext}
	\depedge{2}{1}{SUBJ}
	\depedge{2}{3}{PRED}
	\depedge{5}{4}{ENDP}
	\depedge[edge unit distance=1.8ex]{2}{5}{END}
	\depedge{5}{6}{NEG}
	\depedge[edge unit distance=2.5ex]{5}{7}{PRED}
	\depedge[edge unit distance=1.5ex]{2}{8}{PUNCT}
	\deproot[edge unit distance=4ex]{2}{ROOT}
\end{dependency}
% \captionof{figure}{CHILDES}
\end{minipage}
\captionof{figure}{CHILDES: tag questions}
\label{fig:taggr}


It is clear from the dependency graphs that the CHILDES dependency trees are broken. Therefore, to produce a proper dependency tree from that, one must change the head index of the token identified as the fake root (indicated by the GR label \texttt{BEG}) to that of the sentence's true root.

After the change-head operation is applied, the current token with GR label \texttt{BEG} is connected to the only one root remained. Now the two examples in \Cref{fig:intj} and \Cref{fig:voc} share the same dependency structures. However, dependency relation labels for vocatives and interjections should be different. To make the final decision, chatconllu needs POS tag of the token with \texttt{BEG}. The decision process is outlined below:\\

\begin{itemize}
	\item If the token has a POS tag and it is not \texttt{INTJ}, \texttt{PROPN}, or \texttt{NOUN}, the corresponding \texttt{deprel} \sidenote{Definition for \texttt{parataxis} can be found from \url{https://universaldependencies.org/u/dep/parataxis.html}} label should be \texttt{parataxis} for a loose side-by-side placement of the affected elements.
	\item If the POS tag is \texttt{INTJ}, then the first part of the sentence is considered as an exclamation. The corresponding UD-style label should be \texttt{discourse} \sidenote{Definition for \texttt{discourse} can be found from \url{https://universaldependencies.org/u/dep/discourse.html}}.
	\item If the POS tag is \texttt{PROPN} or \texttt{NOUN}, the first part of the sentence is considered as an expression in vocative. Therefore, the corresponding \texttt{deprel} label should be \texttt{vocative} \sidenote{Definition for \texttt{vocative} can be found from \url{https://universaldependencies.org/u/dep/vocative.html}}.
\end{itemize}

% For instance, the sentence "well, son, that's just life, isn't it ?" will be analysed by GRASP as this:
\subsection{Inconsistent Dependency Parses}

When searching for cases of tag questions, I noticed that the same tag question are parsed differently by GRASP. Right now chatconllu cannot change them into UD-style dependency structures automatically, so the structures, although problematic, are kept as they are. Only the dependency labels are na\"{\i}vely changed to their UD counterparts according to the mapping I created.

\subsection{Predicate Relations}

In GRASP's syntactic analysis scheme, the grammatical relation \texttt{PRED} identifies a nominal or adjectival predicate and the predicate is seen as the argument of its head, which is a verb (\cite{Macwhinney2000}, Part 3, p.63). However, in UD-style dependency annotations, predicate with copula verb constructions favors the predicate as the head, the differences can be seen from the two parses of the same sentence in \Cref{fig:copula}.\\

\begin{minipage}[b]{0.5\linewidth}
\begin{dependency}
	\begin{deptext}[column sep=0.5cm]
	She \& is \& a \& student \& .\\
	% PRON \& AUX \& DET \& NOUN \& PUNCT\\
	\end{deptext}
	\depedge{2}{1}{SUBJ}
	\depedge{2}{4}{PRED}
	\depedge{4}{3}{DET}
	\depedge{2}{5}{PUNCT}
	\deproot{2}{ROOT}
\end{dependency}
\end{minipage}
\hfill
\begin{minipage}[b]{0.5\linewidth}
\begin{dependency}
	\begin{deptext}[column sep=0.5cm]
	She \& is \& a \& student \& .\\
	% PRON \& AUX \& DET \& NOUN \& PUNCT\\
	\end{deptext}
	\depedge{4}{1}{nsubj}
	\depedge{4}{2}{aux}
	\depedge{4}{3}{det}
	\depedge{4}{5}{punct}
	\deproot{4}{root}
\end{dependency}
\end{minipage}\captionof{figure}{CHILDES vs UD: nominal predicate with copula verbs}\label{fig:copula}
\clearpage

However, constructions of predicate with linking verbs, like the example in \Cref{fig:become} with the verb \emph{become}, receive the same dependency structure.\\

\begin{minipage}[b]{0.5\linewidth}
\begin{dependency}
	\begin{deptext}[column sep=0.5cm]
	She \& became \& happier \& .\\
	% PRON \& VERB \& ADJ \& PUNCT\\
	\end{deptext}
	\depedge{2}{1}{SUBJ}
	\depedge{2}{3}{PRED}
	\depedge{2}{5}{PUNCT}
	\deproot{2}{ROOT}
\end{dependency}
\end{minipage}
\begin{minipage}[b]{0.5\linewidth}
\begin{dependency}
	\begin{deptext}[column sep=0.5cm]
	She \& became \& happier \& .\\
	% PRON \& VERB \& ADJ \& PUNCT\\
	\end{deptext}
	\depedge{2}{1}{nsubj}
	\depedge{2}{3}{xcomp}
	\depedge{2}{5}{punct}
	\deproot{2}{root}
\end{dependency}
\end{minipage}
\captionof{figure}{CHILDES vs UD: predicate with linking verb \emph{become}}
\label{fig:become}
\vspace{0.5em}

In fact, although CHILDES' analyses of copula structures are consistent, always making the copula verb the root, UD treats the cases differently. Three types of copula constructions that produce different UD-style parses are identified with comparison to the CHILDES parses in (\cite{liu2021}):
\begin{itemize}
	\item copula structure with statements like the example in \Cref{fig:copula}
	\item copula structures in Wh- and How-questions:\\ The interrogative words are analysed as the subject of the construction in CHILDES, while in UD-style analysis, the predicate becomes the subject.
	\item copula construction in expletives like \emph{there be} expressions:\\
	CHILDES has \emph{there} analysed as the subject, while in UD-style analysis, the predicate in there-be constructions is the subject.
\end{itemize}

This shows that to obtain UD dependency parses require not only a mapping of the labels used for the grammatical relations but also a case-by-case analysis with the annotation conventions of both formalisms in mind, which presents a problem for automatic conversion.\\

Many other cases that are challenging for adapting CHILDES syntactic annotations to UD-style are discussed and implemented in (\cite{liu2021}), which are not handled by chatconllu. They have made the code for conversion public on GitHub \sidenote{\url{https://github.com/zoeyliu18/Parsing_Speech/tree/main/code}}.\\

Apart from that, Liu and Prud’hommeaux observed that depending on the actual meaning of the child, the utterance \emph{man no spoon} can be interpreted in at least three different ways (\cite{liu2021}). Odijk et al. also reported that there are various reasons for a child to produce speech different from the conventional form used by adult speakers. Sometimes the true meanings are concealed by the same transcriptions, which requires \emph{intensive investigation of each phenomenon} by researchers. It seems that for spoken data like child speech, information like context and tones that help us disambiguate meanings are not conveyed by literal transcriptions. Merely looking at utterance texts, it is even hard for a human to decide the real meaning of the utterance. Even if rules are defined for each grammatical relation, incomplete utterances can still produce unintended results in a decision making process. Therefore, instead of coming up case-based rules to convert the syntactic relation labels and structures, chatconllu stores the original label in the token's \texttt{MISC} field and uses a direct mapping to convert the dependency label while keeping the tree structure produced by GRASP, except for sentences with multiple roots.



\chapter{Validation and Comparison} % Main chapter title

\label{Chapter5} % For referencing the chapter elsewhere, use \ref{Chapter5}

\section{Validating CHAT using CLAN Tools}

After using chatconllu to convert the conllu files back to CHAT format, the user may want to validate that the resulting cha files are well-formed and accepted by the CLAN program. CLAN has a CHECK program \sidenote{CLAN can be downloaded via \url{https://dali.talkbank.org/clan/}} that validates the CHAT transcription format. Alternatively, if the user wants to stick to the command line, they can download Chatter \sidenote{Chatter - CHAT format validator: \url{https://talkbank.org/software/chatter.html}}, a Java program from TalkBank.

% include a use case.

For the chosen corpora, CHAT files produced by chatconllu pass the CHECK program in CLAN but not Chatter for the reason of line-wrapping. Long lines in the original cha files generated from the CLAN program are broken after a certain amount of characters is reached. Continuations of the line are preceded by an initial tab. However, the current version of chatconllu does not handle the maximum length of a line, and because of that, it fails the Chatter validation process.

The differences between the original and back-converted cha files can be visually inspected using \emph{icdiff}. With this command, a side-by-side display of the two cha files selected is displayed in the terminal, with the differences coloured.

\section{Validating CoNLL-U using UD Tools}

UD also provides tools for format validation. One of the UD maintained tools is the script \emph{validate.py} \sidenote{link to file on GitHub: \url{https://github.com/UniversalDependencies/tools/blob/master/validate.py}}. However, what's tricky about UD format validation is that this script also validates the accepted values of each language in UD. To run the script, one has to input a language code. Although there are five levels of validation, the conllu files generated by the current version of chatconllu only pass the first two levels, which does not check if the values are accepted by UD or specifications of the given language.

\section{Related Work}

\paragraph{pyconll} % (fold)
\label{par:pyconll}
\emph{pyconll} (\cite{pyconll}) is a low-level API for processing CoNLL-U formatted files. In this thesis, I used pyconll for parsing the converted conllu files and especially for reading the values of individual fields.

% \paragraph{CHAT2CONLLU and CONLL2CHAT} % (fold)
% \label{par:chat2conllu}
% \emph{CHAT2CONLL} and \emph{CONLL2CHAT} are officially part of the CLAN program that transforms CHAT files to CoNLL formats.

\paragraph{Dependency Parsing for Low-resource Spontaneous Speech} % (fold)
\label{par:zoey}
(\cite{liu2021}) used a low-resource setting, focusing on one child from the Brown Corpus (\cite{brown1973}) and tested the performance of dependency parsers on parent-child conversations. The out-domain parser was trained on written text and performed well on adult speech but not so well on child speech, while the parser trained on a limited amount of in-domain spoken data improved the parsing result of child speech to be comparable with adult speech. As part of their study, they created a semi-automatic adaptation of annotation formats from CHILDES to UD.

\paragraph{The AnnCor CHILDES Treebank}
(\cite{odijk2018anncor}) preprocessed the Dutch CHILDES corpora, augmented them with syntactic annotations using the Alpino parser, and performed partial manual verification. While creating this treebank, they noticed the many disadvantages of annotation discrepancies and argued that annotation guidelines should be explicit and should be strictly followed to produce consistency. They suggest combining human annotations with automatic checks by computer programs. They also pointed out that although the grammar used in child speech is limited since the child is still acquiring the language, assigning adult syntactic structures to their utterances is still valid.

\chapter{Application} % Main chapter title

\label{Chapter6} % For referencing the chapter elsewhere, use \ref{Chapter6}

% What are the potential ways chatconllu can be used?
As a two-way converter, chatconllu can be used as an intermediate step to augment information to either annotation format. For instance, one can start with cha files, convert them to conllu files and pass those files to UDPipe for dependency parsing. After that, the files can either be exported as a preliminary speech treebank or converted back to CHAT format with added dependent tiers.

% \chapter{Ideas} % Main chapter title

\label{Chapter7} % For referencing the chapter elsewhere, use \ref{Chapter7}

% motivation?
Unlike most written text or lab speech data, spontaneous speech is <chaotic> in form. Without drafting or editing, a sentence settle in form as each word is spoken. Errors and markers of disfluency like pauses and repetitions are ubiquitous in spontaneous speech data. Before a consensus is reached in how people transcribe and annotate speech, transcribers had to device their own markings/symbols to faithfully transcribe the utterances/conversational interactions. The disadvantages are clear, inconsistent use of symbols is an obstacle for the sharing of linguistic data, which is indispensable in Natural Language Processing.\\

% references
CHILDES database \cite{Macwhinney2000}\\
the Brown Corpus <needs bibtex (Brown, 1973)>\\

\chapter{Ideas} % Main chapter title

\label{Chapter8} % For referencing the chapter elsewhere, use \ref{Chapter8}

\section{Morphosyntax in CHAT and CoNLL-U}

Both annotation formalisms support the storage of morphosyntactic information, although in different ways. In this chapter, I show the general structure of these formats using minimal examples, compare their ways of organising information and point out issues that I found worth discussing during the implementation process of chatconllu.

\subsection{Extracting Information from Dependent Tiers}



\paragraph{lemma} After looking at the syntax of the \%mor tier, one should realise that getting the lemma out of a MOR string is not an easy task. There are several pitfalls to avoid:\\
\begin{itemize}
	\item The semantics of the symbols may be ambiguous. Take the dash symbol as an example, for dashed words (words with dash symbols) like \texttt{tic-tac-toe}, it is merely a connector between parts of the word, but it could also appear as an indicator for suffixes, as in the example of \texttt{part|get-PRESP} for the word \emph{getting}.
	\item One should also notice that the stem is not the lemma, just that for English, these two share the same form in most cases. For instance, for \emph{untied}, we have the MOR segment \texttt{un\#v|tie-PASTP}, which means in plain English that the word has a stem \emph{tie} with prefix \emph{un}, its part-of-speech is verb and this word form is the past participle of the verb. On the other hand, in CoNLL-U, the lemma field is reserved for lemma, not word stem, therefore the prefix should be put back in place to produce the real lemma.
\end{itemize}


%----------------------------------------------------------------------------------------
% THESIS CONTENT - APPENDICES
%----------------------------------------------------------------------------------------

% By using input instead of include for the chapters we are able to move the following line here
% Therefore the addition before the last chapter is not necessary anymore.

% Call the following chapters "Appendix" inside the table of contents
\addtocontents{toc}{\string\def\string\chaptername{Appendix}}

\appendix % Cue to tell LaTeX that the following "chapters" are Appendices

% Ensure proper section numbering in appendix, e.g., A.1, A.2, B.1, …
\renewcommand{\thesection}{\thechapter.\arabic{section}}
\renewcommand{\thesubsection}{\thesection.\arabic{subsection}}
\renewcommand{\thesubsubsection}{\thesubsection.\arabic{subsubsection}}

%%% CHANGES NEEDED HERE
%
% Include the appendices of the thesis as separate files from the Appendices folder
% Uncomment the lines as you write the Appendices

% Appendix A
 
\chapter{Morphosyntactic Mappings}
\label{appendixa}

\section{Part-of-Speech Mapping}

\begin{table}[h!]
\caption {Complete Set of Universal POS Tags} \label{tab:uposset}
\begin{tabular}{@{}lllll@{}}
\toprule
\textbf{Universal POS Tags} & \textbf{Description}\\ \midrule
ADJ & adjective\\
ADV & adverb\\
INTJ & interjection\\
NOUN & noun\\
PROPN & proper noun\\
VERB & verb\\\midrule
ADP & adposition\\
AUX & auxiliary verb\\
CCONJ & coordinating conjunction\\
DET & determiner\\
NUM & numeral\\
PART & particle\\
PRON & pronoun\\
SCONJ & subordinating conjunction\\\midrule
PUNCT & punctuation\\
SYM & symbol\\
X & other\\\bottomrule
\end{tabular}\\
\vspace{0.5cm}\\
All UPOS tags except \texttt{SYM} are used in chatconllu's POS mappings.\\
\end{table}

\newpage
\begin{table}[htp!]
\caption {chatconllu Part-of-Speech Mapping 1} \label{tab:posmap1}
\centering
\begin{tabularx}{\linewidth}{@{}lXl@{}}
\toprule
\textbf{MOR POS Code} & \textbf{Description} & \textbf{UPOS}\\ \midrule
adj & adjective & ADJ\\
adj:pred & adjective-predicative & ADJ\\\midrule
adv & adverb & ADV\\
adv:tem & adverb-temporal & ADV\\
neg & negative & ADV\\\midrule
co & communicator & INTJ\\
fil & filler & INTJ\\
wplay & wordplay & INTJ\\
bab & babbling & INTJ\\
sing & singing & INTJ\\\midrule
qn & quantifier & DET\\
det:art & determiner-article & DET\\
det:dem & determiner-demonstrative & DET\\
det:int & determiner-interrogative & DET\\
det:num & determiner-numeral & DET\\
det:poss & determiner-possessive & DET\\
quant & quantifier & DET\\
art & article & DET\\
prepart & preposition with article & DET\\\midrule
aux & auxiliary & AUX\\
v:aux & verb-auxiliary & AUX\\
cop & verb-copula & AUX\\
mod & verb-modal & AUX\\\midrule
post & postmodifier & ADP\\
prep & preposition & ADP\\
vpfx & preverb/verbal particles & ADP\\\midrule
part & particle & PART\\
inf & infinitive & PART\\\midrule
coord & coordinator & CCONJ\\\midrule
conj & conjunction & SCONJ\\
comp & complementizer & SCONJ\\\midrule
num & numeral & NUM\\\bottomrule
\end{tabularx}\\
\vspace{0.5cm}
All MOR part-of-speech categories handled by the current version of chatconllu (part 1).\\
\end{table}

\newpage
\vspace{-2cm}
\begin{table}
\caption {chatconllu Part-of-Speech Mapping 2} \label{tab:posmap2}
\centering
\begin{tabularx}{\linewidth}{@{}lXl@{}}
\toprule
\textbf{MOR POS Code} & \textbf{Description} & \textbf{UPOS}\\ \midrule
n & noun & NOUN\\
n:let & noun-letter & NOUN\\
n:pt & noun-plurale tantum & NOUN\\
on & onomatopoeia & NOUN\\
onoma & onomatopoeia & NOUN\\\midrule
pro:dem & pronoun-demonstrative & PRON\\
pro:exist & pronoun-existential & PRON\\
pro:indef & pronoun-indefinitive & PRON\\
pro:int & pronoun-interrogative & PRON\\
pro:obj & pronoun-object & PRON\\
pro:per & pronoun-personal & PRON\\
pro:poss & pronoun-possessive & PRON\\
pro:refl & pronoun-reflexive & PRON\\
pro:rel & pronoun-relative & PRON\\
pro:sub & pronoun-subject & PRON\\\midrule
n:prop & proper noun & PROPN\\\midrule
v & verb & VERB\\
vimp & verb imperative & VERB\\\midrule
end & MOR punctuation mark & PUNCT\\
beg & MOR punctuation mark & PUNCT\\
cm & MOR punctuation mark & PUNCT\\
bq & MOR punctuation mark & PUNCT\\
eq & MOR punctuation mark & PUNCT\\
bq2 & MOR punctuation mark & PUNCT\\
eq2 & MOR punctuation mark & PUNCT\\\midrule
none & none & X\\
dia & dialect & X\\
test & test words like "wug" & X\\
meta & metalinguistic use & X\\
phon & phonologically consistent form & X\\
fam & family-specific form & X\\
uni & Unibet transcription & X\\
L2 & second-language form & X\\
neo & neologism & X\\
chi & child-invented form & X\\\bottomrule
\end{tabularx}\\
\vspace{0.5cm}
All MOR part-of-speech categories handled by the current version of chatconllu (part 2).\\
\end{table}
\clearpage

\newpage
\section{Morphological Features Mapping}
\begin{table}[h!]
\caption {chatconllu Morpheme-Feature Mapping} \label{tab:featsmap}
\centering
\begin{tabularx}{\linewidth}{@{}lXl@{}}
\toprule
% \caption {chatconllu Morpheme-Feature Mapping} \label{tab:featsmap}
% \begin{tabular}{@{}lll@{}}
% \toprule
\textbf{Morpheme Codes} & \textbf{UD feats}\\ \midrule
	12s & Number=Sing|Person=2\\
	13s & Number=Sing|Person=3\\
	1p & Number=Plur|Person=1\\
	1s & Number=Sing|Person=1\\
	23s & Number=Sing|Person=3\\
	2p & Number=Plur|Person=2\\
	2s & Number=Sing|Person=2\\
	3p & Number=Plur|Person=3\\
	3s & Number=Sing|Person=3\\
	3sp & Number=Plur|Person=3\\
	cond & Mood=Cnd\\
	cp & Degree=Cmp\\
	f & Gender=Fem\\
	fem & Gender=Fem\\
	fut & Tense=Fut\\
	inf & VerbForm=Inf\\
	imp & Mood=Imp\\
	impf & Tense=Imp\\
	m & Gender=Masc\\
	pass & Voice=Pass\\
	past & Tense=Past\\
	pastp & Tense=Past|VerbForm=Part\\
	pl & Number=Plur\\
	poss & Poss=Yes\\
	pp & Tense=Past|VerbForm=Part\\
	ppre & Tense=Pres|VerbForm=Part\\
	pres & Tense=Pres\\
	presp & Tense=Pres|VerbForm=Part\\
	pret & Tense=Past\\
	sg & Number=Sing\\
	sp & Degree=Sup\\
	sub & Mood=Sub\\
	subj & Mood=Sub\\\bottomrule
\end{tabularx}\\
\vspace{0.5cm}
All MOR grammatical morpheme codes handled by the current version of chatconllu.\\
\end{table}


\newpage
\section{Dependency Relation Mapping}
\begin{table}[h!]
\caption {chatconllu Dependency Relation Mapping} \label{tab:grmap}
\centering
\begin{tabularx}{\linewidth}{@{}lXl@{}}
\toprule
\textbf{Morpheme Codes} & \textbf{UD feats}\\ \midrule
	12s & Number=Sing|Person=2\\
	13s & Number=Sing|Person=3\\
	1p & Number=Plur|Person=1\\
	1s & Number=Sing|Person=1\\
	23s & Number=Sing|Person=3\\
	2p & Number=Plur|Person=2\\
	2s & Number=Sing|Person=2\\
	3p & Number=Plur|Person=3\\
	3s & Number=Sing|Person=3\\
	3sp & Number=Plur|Person=3\\
	cond & Mood=Cnd\\
	cp & Degree=Cmp\\
	f & Gender=Fem\\
	fem & Gender=Fem\\
	fut & Tense=Fut\\
	inf & VerbForm=Inf\\
	imp & Mood=Imp\\
	impf & Tense=Imp\\
	m & Gender=Masc\\
	pass & Voice=Pass\\
	past & Tense=Past\\
	pastp & Tense=Past|VerbForm=Part\\
	pl & Number=Plur\\
	poss & Poss=Yes\\
	pp & Tense=Past|VerbForm=Part\\
	ppre & Tense=Pres|VerbForm=Part\\
	pres & Tense=Pres\\
	presp & Tense=Pres|VerbForm=Part\\
	pret & Tense=Past\\
	sg & Number=Sing\\
	sp & Degree=Sup\\
	sub & Mood=Sub\\
	subj & Mood=Sub\\\bottomrule
\end{tabularx}\\
\vspace{0.5cm}
All GRASP GRs handled by the current version of chatconllu.\\
\end{table}

% \include{appendices/appendixB}
% \include{appendices/appendixC}



%----------------------------------------------------------------------------------------
% BIBLIOGRAPHY
%----------------------------------------------------------------------------------------

% Bibliography has no wide margins:
\newgeometry{
    inner=2cm, % Inner margin
    outer=2cm, % Outer margin
    marginparwidth=0cm,
    marginparsep=0mm,
    bindingoffset=.5cm, % Binding offset
    top=1.5cm, % Top margin
    bottom=2.5cm, % Bottom margin,
    includehead,
    includefoot
    % showframe, % Uncomment to show how the type block is set on the page
}

\addchap{References}

% enables two-column layout for bibliography
\setlength\columnsep{2em}
\begin{multicols}{2}
    \begin{refcontext}[sorting=nyt] % sort bibliography by last name, year, title
        \renewcommand*{\bibfont}{\small\RaggedRight}
        \linespread{1.0}\selectfont % increase linespread if desired (not recommended)
        \printbibliography[heading=none]
    \end{refcontext}
\end{multicols}

%----------------------------------------------------------------------------------------

%----------------------------------------------------------------------------------------
% DECLARATION PAGE
%----------------------------------------------------------------------------------------

\begin{declaration}
\addchaptertocentry{\authorshipname} % Add the declaration to the table of contents

% TODO Change the declaration according as needed. *

%\selectlanguage{ngerman}
Ich erkläre hiermit gemä\ss\ \S~17 Abs.\,2 APO, dass ich die vorstehende {\thesistype}arbeit selbständig\\ verfasst und keine anderen als die angegebenen Quellen und Hilfsmittel benutzt habe.

\bigskip
\bigskip

\begin{tabular}{@{}l@{}}
    T\"ubingen, den \rule[-0.8em]{10em}{0.5pt}\\[2ex]
    ~
\end{tabular}
\hspace{\fill}%
\begin{tabular}{@{}c@{}}
    \rule[-0.8em]{20em}{0.5pt}\\[2ex]
    \authorname
\end{tabular}\hspace{\fill}

\end{declaration}

\end{document}
