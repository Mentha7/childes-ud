%%%%%%%%%%%%%%%%%%%%%%%%%%%%%%%%%%%%%%%%%%%%%%%%%%%
%
% File: setup.tex
%
% This file is part of the PSIThesis.cls
% LaTeX documentclass
%
% The code in this file is made available
% under the following license:
%
% LPPL v1.3c (http://www.latex-project.org/lppl)
%
%%%%%%%%%%%%%%%%%%%%%%%%%%%%%%%%%%%%%%%%%%%%%%%%%%%


%----------------------------------------------------------------------------------------
%   UNIVERSITY OF BAMBERG COLORS
%----------------------------------------------------------------------------------------

% http://www.brandwares.com/RGBTintCalculator.php
% Base Color value obtained from UB Corporate Identity Manual
\definecolor{ubblue}{HTML}{00457D}
\definecolor{ubblue80}{HTML}{336A97}
% \definecolor{ubblue60}{HTML}{668FB1}
% \definecolor{ubblue40}{HTML}{99B5CB}
% \definecolor{ubblue20}{HTML}{CCDAE5}

\definecolor{ubyellow}{HTML}{FFD300}
\definecolor{ubyellow25}{HTML}{FFF4BF}

\definecolor{ubred}{HTML}{e6444F}

\definecolor{ubgreen}{HTML}{97BF0D}

\definecolor{gray75}{gray}{0.75}
\definecolor{gray50}{gray}{0.50}

% https://imagecolorpicker.com/color-code/a11e3b
% Base Color value obtained from Universitaet Tuebingen Logo
\definecolor{utred}{HTML}{A11E3B}
\definecolor{utred80}{HTML}{81182F}
\definecolor{utred60}{HTML}{611223}
\definecolor{utred40}{HTML}{400C18}
\definecolor{utred20}{HTML}{20060c}
\definecolor{triadicblue}{HTML}{1E3BA1}

%----------------------------------------------------------------------------------------
%   FONT SETUP
%----------------------------------------------------------------------------------------

%\usepackage[T1]{fontenc} % Output font encoding for international characters - do not use T1 encoding with luatex!
\usepackage[utf8]{luainputenc} % makes unicode characters like –, €, and ß work properly

% amssymb must be loaded before newtxmath to avoid this error:
% Command `\Bbbk' already defined
\usepackage{amssymb}
\usepackage[cochineal]{newtxmath} % must be loaded before fontspec

\usepackage[no-math]{fontspec} % allows us to use OTF/TTF fonts, but do not interfere with math (because we use newtxmath, which does support Cochineal)

% If you cannot use the cochineal font, uncomment the following lines to select
% the Crimson font. Note, however, that you'll have to take care of the math font
% on your own.
%
%\setmainfont[
% Path           = fonts/,
%    BoldFont       = {Crimson-Semibold.otf},
%    ItalicFont     = {Crimson-Italic.otf},
%    BoldItalicFont = {Crimson-BoldItalic.otf}
%]{Crimson-Roman.otf}

\setmainfont{Cochineal}[
  Numbers={Proportional,OldStyle},
  Style=Swash % for nice swashed Q letter, see https://golatex.de/spezeielle-opentype-features-in-fontspec-aktivieren-t19831.html
]
%\setmainlanguage{english}
\DeclareSymbolFont{operators}{\encodingdefault}{\familydefault}{m}{n} %  render numbers in cochineal, cf. https://tex.stackexchange.com/questions/398895/using-two-different-math-fonts-with-lualatex

% imitate the behavior of the cochineal package as follows:
% cf. https://tex.stackexchange.com/questions/448895/fontenc-vs-fontspec-with-xelatex
\DeclareRobustCommand{\lfstyle}{\addfontfeatures{Numbers=Lining}}
\DeclareTextFontCommand{\textlf}{\lfstyle}
\DeclareRobustCommand{\tlfstyle}{\addfontfeatures{Numbers={Tabular,Lining}}}
\DeclareTextFontCommand{\texttlf}{\tlfstyle}

% Exception: tables should use "lining figures" (all digits having same width)
\AtBeginEnvironment{tabular}{%
  \tlfstyle
}
\AtBeginEnvironment{tabularx}{%
  \tlfstyle
}


% monospace font, will be used in verbatim and listing environments
\setmonofont[
  Path           = fonts/,
    BoldFont       = {iosevka-ss04-bold.ttf},
    ItalicFont       = {iosevka-ss04-italic.ttf},
    BoldItalicFont       = {iosevka-ss04-bolditalic.ttf},
    Scale = 0.83 % manually determined value;
]{iosevka-ss04-regular.ttf}


% sans-serif font, will be used in the margins
\setsansfont[
  Path            = fonts/,
  BoldFont    = Roboto-Bold.otf,
  ItalicFont    = Roboto-Italic.otf,
  BoldItalicFont  = Roboto-BoldItalic.otf,
  Scale = 0.83 % manually determined value;
]{Roboto-Regular.otf}

\renewcommand{\familydefault}{\rmdefault}
\defaultfontfeatures{Ligatures=TeX}


%----------------------------------------------------------------------------------------
%   HEADINGS SETUP (CHAPTERS, SECTIONS, …)
%----------------------------------------------------------------------------------------

\usepackage[explicit]{titlesec}
\newcommand{\hsp}{\hspace{20pt}}

\setcounter{secnumdepth}{3}

% We use lining figures for headers (tlfstyle) because they fit better with uppercase letters than old-style figures.

% chapters have a vertical line between number and title
\titleformat{\chapter}[hang]{\Huge\bfseries\tlfstyle}{\color{black}\thechapter}{20pt}{\begin{tabular}[t]{@{\color{utred}\vrule width 2pt\hsp}p{0.85\textwidth}}\raggedright #1\end{tabular}}

% sections
\titleformat{\section}[hang]{\bfseries\large\tlfstyle}{{\color{utred60}\thechapter.\arabic{section}}}{1ex}{{\color{utred60} #1}}{}

% subsections
\titleformat{\subsection}[hang]{\bfseries\large\tlfstyle}{{\color{utred80}\thechapter.\arabic{section}.\arabic{subsection}}}{1ex}{{\color{utred80} #1}}{}

% subsubsections
\titleformat{\subsubsection}[hang]{\bfseries\tlfstyle}{{\color{utred}\thechapter.\arabic{section}.\arabic{subsection}.\arabic{subsubsection}}}{1ex}{{\color{utred} #1}}{}

% vertical spacing for headings ==============
\titlespacing*{\section}
{0pt}{7ex}{3ex}

\titlespacing*{\subsection}
{0pt}{4ex}{2ex}

\titlespacing*{\subsubsection}
{0pt}{4ex}{2ex}
% end of vertical spacing ====================

%----------------------------------------------------------------------------------------
%   TABLE OF CONTENTS SETUP
%----------------------------------------------------------------------------------------

% solution inspired from https://tex.stackexchange.com/questions/178510/how-can-i-reproduce-this-beautiful-table-of-contents
\usepackage{etoc}
\etocsetlevel{section}{2}
\etocsetlevel{subsection}{3}

\etocsettocdepth{section} % set to subsection for adding subsections to toc (not recommended)

\newlength{\tocleft}
\setlength{\tocleft}{2.5cm} % must be set to fit the innermargin defined in geometry (change only if you have changed the margins)

\newlength{\tocsep}
\setlength{\tocsep}{2em}

\usepackage{textcase}


\etocsetstyle{chapter}
   {}
   {}
   {\etocifnumbered
     {\makebox[0pt][r]
       % we use \etocthenumber instead of \etocnumber to avoid the href, which is part of \etocthenumber, messing with MakeTextLowercase
       {\textsc{\MakeTextLowercase\chaptername\ \MakeTextLowercase\etocthenumber}\hspace{\tocsep}}%
       \textbf{\etocname\kern1em\relax\etocpage}%
    }%
    {\textbf{\etocname\kern1em\relax\etocpage}}%
    \par\vspace{3ex}%
   }%
   {}

\etocsetstyle{section}
   {\vspace{-2ex}} % Muss von den 3ex aus Chapter abgezogen werden
   {}
   % see the comment regarding etocthenumber in the chapter style definition
   {\makebox[0pt][r]{\textsc{\MakeTextLowercase\etocthenumber}\hspace{\tocsep}}%
    \etocname\kern1em\etocpage\par%
   }%
   {\addvspace{3ex}} % 3ex falls danach Chapter kommt

\etocsetstyle{subsection}
   {\vspace{0ex}}
   {}
   {\makebox[3em][l]{\etocnumber}\etocname\kern1em\etocpage\par}
   {\addvspace{2ex}} % 2ex falls danach Section kommt

\etocsettocstyle{\chapter*{\contentsname}
                \thispagestyle{plain}%
                \leftskip\tocleft\parindent0pt}{}


%----------------------------------------------------------------------------------------
%   OTHER PACKAGES
%----------------------------------------------------------------------------------------
\usepackage{tikz-dependency}  % for dependency trees
\usepackage{tabularx} % for more flexible tables

\usepackage{marginnote} % Enable Notes on the Page Margin
\usepackage{marginfix} % Enables floating for margin figures
\usepackage{ragged2e} % provides better hyphenation, use with camel case: \RaggedRight
\renewcommand*{\raggedleftmarginnote}{\RaggedLeft}
\renewcommand*{\raggedrightmarginnote}{\RaggedRight}

% justified margin notes:
% uncomment the following two lines to for justified layout of margin notes
%\renewcommand*{\raggedleftmarginnote}{\RaggedLeft}
%\renewcommand*{\raggedrightmarginnote}{\RaggedRight}



\renewcommand*{\marginfont}{\setlength{\parskip}{0.5ex}\scriptsize\sffamily} % format margin text



% for sidenotes: change marginpar font
\usepackage{xparse}
\let\oldmarginpar\marginpar
\RenewDocumentCommand{\marginpar}{om}{%
  \IfNoValueTF{#1}
    {\oldmarginpar{\mymparsetup #2}}
    {\oldmarginpar[\mymparsetup #1]{\mymparsetup #2}}}
\newcommand{\mymparsetup}{\scriptsize\sffamily}


% this provides correct alignment for margin text that is inserted
% right at the beginning of a paragraph; however, it messes up the
% alignment in all other cases.
%% therefore, removed for now:
%%\renewcommand{\marginnotevadjust}{0.71\baselineskip}
% The following is the necessary correction for in-paragraph use
\renewcommand{\marginnotevadjust}{0.21\baselineskip}
\renewcommand{\marginnotevadjust}{0.55\baselineskip}

\usepackage{microtype} % enable better typographic setup

\usepackage{multicol} % enable usage of multiple columns

% biblatex setup
% inspired by https://anneurai.net/2017/10/18/thesis-formatting-in-latex/
\usepackage[
  backend=biber,
  style=authoryear-comp,
  maxcitenames=2,
  maxbibnames=4, % when does the citation change to et al?
  uniquelist=false,
  uniquename=false,
  firstinits=true, % author initials in list
  terseinits=true, % no points between initials
  giveninits=true, % always print only initials for given names
  doi=false, % these fields are commonly omitted
  isbn=false, % these fields are commonly omitted
  url=false, 
  dashed=false,
  terseinits=false, % true if no points between initials
  sortcites=true,
  sorting=ynt,
  language=english,
  backref=true, % show on what pages a ref has been cited
]{biblatex} % Use biber backend with alphabetic reference style
\AtEveryBibitem{%
  \clearlist{language} % don't show "en."
  \clearlist{extra} % clears extra fields such as ISBN nrs
}

% shorten the strings used in back references
\DefineBibliographyStrings{english}{%
  backrefpage = {page},
  backrefpages = {pages},
}

%-- no "quotes" around titles of chapters/article titles
\DeclareFieldFormat[article, inbook, incollection, inproceedings, misc, thesis, unpublished]{title}{#1}
%-- no punctuation after volume
\DeclareFieldFormat[article]{volume}{{#1}}
%-- puts number/issue between brackets
\DeclareFieldFormat[article, inbook, incollection, inproceedings, misc, thesis, unpublished]{number}{\mkbibparens{#1}}
%-- and then for articles directly the pages w/o any "pages" or "pp."
\DeclareFieldFormat[article]{pages}{#1}
%-- format 16(4):224--225 for articles
\renewbibmacro*{volume+number+eid}{\printfield{volume}\printfield{number}\printunit{\addcolon}
}
\DeclareFieldFormat{url}{\url{#1}}


\usepackage[autostyle=true]{csquotes} % Required to generate language-dependent quotes in the bibliography

\usepackage[
  obeyFinal,
  textsize=scriptsize,
  backgroundcolor=ubyellow25,linecolor=ubyellow,bordercolor=ubyellow,
]{todonotes}

% change font of todo notes to sans-serif
\makeatletter
\renewcommand{\todo}[2][]{\@bsphack\@todo[#1]{\sffamily #2}\@esphack\ignorespaces}
\makeatother

\usepackage{booktabs} % use formal table layout

\urlstyle{same} % avoids printing URLs in typewriter font

% enable very extensive URL breaking
% https://tex.stackexchange.com/questions/3033/forcing-linebreaks-in-url
\PassOptionsToPackage{hyphens}{url}
\expandafter\def\expandafter\UrlBreaks\expandafter{\UrlBreaks% save the current one
  \do\a\do\b\do\c\do\d\do\e\do\f\do\g\do\h\do\i\do\j%
  \do\k\do\l\do\m\do\n\do\o\do\p\do\q\do\r\do\s\do\t%
  \do\u\do\v\do\w\do\x\do\y\do\z\do\A\do\B\do\C\do\D%
  \do\E\do\F\do\G\do\H\do\I\do\J\do\K\do\L\do\M\do\N%
  \do\O\do\P\do\Q\do\R\do\S\do\T\do\U\do\V\do\W\do\X%
  \do\Y\do\Z\do\*\do\-\do\~\do\'\do\"\do\-}%

% TODO consider using package xurl, which is supposed to handle url breaking

% https://tex.stackexchange.com/a/450695
% allow URLs to be spaced out at / => much better URL breaking in margins
\makeatletter
\g@addto@macro\UrlSpecials
{%
    \do\/{\mbox{\UrlFont/}\hskip 0pt plus 0.1em minus 0.1em}%
}
\Urlmuskip=0mu plus 1mu\relax
\makeatother


% hyperlink layout
\usepackage{hyperref}
 \hypersetup{colorlinks,breaklinks,unicode,
             citecolor=triadicblue,
             linkcolor=triadicblue,
             urlcolor=triadicblue,
             filecolor=triadicblue,}

% cleverref allows you to use \Cref{sec:foo} to get the text "Section 1.2".
% This also works with figures and tables.
\usepackage{cleveref}

% datetime is use to automatically handle the date rendering on the titlepage.
\usepackage{datetime}
% rendering the current date as Month/JJJJ
% see: https://tex.stackexchange.com/questions/212263/month-year-format-in-latex
\newdateformat{monthyeardate}{%
  \monthname[\THEMONTH] \THEYEAR}


\raggedbottom % do NOT force all pages to have the same height (which would be done by increasing the space between paragraphs, which can create noisy layouts)

%----------------------------------------------------------------------------------------
% SETUP BIBLIOGRAPHY
%----------------------------------------------------------------------------------------
\setlength{\bibitemsep}{.3\baselineskip plus .05\baselineskip minus .05\baselineskip}
\newlength{\bibparskip}\setlength{\bibparskip}{0pt}
\let\oldthebibliography\thebibliography
\renewcommand\thebibliography[1]{%
  \oldthebibliography{#1}%
  \setlength{\parskip}{\bibitemsep}%
  \setlength{\itemsep}{\bibparskip}%
}

% allow much more liberal line breaks in URLs
\setcounter{biburllcpenalty}{7000}
\setcounter{biburlucpenalty}{8000}

% adjust space between key and entry, default is 2\labelsep
\setlength{\biblabelsep}{1\labelsep}

% configures indentation of bibentries
\defbibenvironment{bibliography}
  {\list
     {\hspace{0.5\labelalphawidth}\bfseries\printtext[labelalphawidth]{%
        \printfield{prefixnumber}%
        \printfield{labelalpha}%
        \printfield{extraalpha}}}
     {\setlength{\labelsep}{\biblabelsep}%
      \setlength{\leftmargin}{0.5\labelalphawidth}%
      \setlength{\itemsep}{1.5\bibitemsep}%
      \setlength{\parsep}{\bibparsep}}%
      \renewcommand*{\makelabel}[1]{##1\hss}}
  {\endlist}
  {\item}


%----------------------------------------------------------------------------------------
% MARGIN SETTINGS
%----------------------------------------------------------------------------------------

% Using the layout from kaobook
\geometry{
    paper=a4paper,
    head=13.6pt,
    top=27.4mm,
    bottom=27.4mm,
    inner=24.8mm,
    %outer=24.8mm,
    %right=2.183cm,
    textwidth=107mm,
    marginparsep=8.2mm,
    marginparwidth=49.4mm,
    %textheight=49\baselineskip,
    includemp,
    % showframe
}

% Wide figures span text and margin.
% Use the pre-calculated length \widefigurewidth in \includegraphics.
\def\widefigurewidth{\dimexpr(\marginparwidth + \textwidth + \marginparsep)}


%----------------------------------------------------------------------------------------
% SETUP HEADER AND FOOTER
%----------------------------------------------------------------------------------------


\newlength{\overflowingheadlen}
\setlength{\overflowingheadlen}{\textwidth}
\addtolength{\overflowingheadlen}{\marginparsep}
\addtolength{\overflowingheadlen}{\marginparwidth}

% old header/footer, maybe not necessary any more?
\automark[chapter]{chapter}
\ihead{\textup{\headmark}} % Inner header; do not use italics: therefore textup
\ihead{\textup{\textsc{\MakeLowercase\headmark}}}% Inner header - use this line for Small Caps in header
\ohead[]{\pagemark} % Outer header
\cfoot[\pagemark]{} % On chapter opening pages, the page number goes centered into the footer
\automark*[section]{}%

% new header/footer, from kaobook; we could probably remove the original definitions from the cls
\renewpagestyle{thesis}{
  {\hspace{-\marginparwidth}\hspace{-\marginparsep}\makebox[\overflowingheadlen][l]{\textup{\thepage}\quad\rule[-\dp\strutbox]{1pt}{\baselineskip}\quad{}\textup{\textsc{\MakeLowercase \leftmark}}}}%
  {\makebox[\overflowingheadlen][r]{\textup{\textsc{\MakeLowercase \rightmark}}\quad\rule[-\dp\strutbox]{1pt}{\baselineskip}\quad\textup{\thepage}}}%
  {}
}{
  {}%
  {}%
  {}
}
\renewpagestyle{plain.thesis}{
  {}%
  {}%
  {}
}{
  {\hspace{-\marginparwidth}\hspace{-\marginparsep}\makebox[\overflowingheadlen][l]{\textup{\textsc{\thepage}}\quad\rule[-\dp\strutbox]{1pt}{\baselineskip}}}%
  {\makebox[\overflowingheadlen][r]{\rule[-\dp\strutbox]{1pt}{\baselineskip}\quad\textup{\textsc{\thepage}}}}%
  {}
}

%----------------------------------------------------------------------------------------
% LISTINGS SETTINGS
%----------------------------------------------------------------------------------------

\usepackage{textcomp}
\usepackage{listings}
\definecolor{darkgray}{rgb}{.4,.4,.4}

\lstdefinelanguage{JavaScript}{
  keywords={typeof, new, true, false, catch, function, return, null, catch, switch, var, if, in, while, do, else, case, break},
  ndkeywords={class, export, boolean, throw, implements, import, this},
  sensitive=false,
  comment=[l]{//},
  morecomment=[s]{/*}{*/},
  morestring=[b]',
  morestring=[b]"
}

\lstset{
    aboveskip={1\baselineskip},
    abovecaptionskip=-1\baselineskip,
    belowcaptionskip=2ex,
    basicstyle=\footnotesize\ttfamily\linespread{4},
    breaklines=true,
    columns=flexible,
    commentstyle=\color{gray50}\ttfamily\itshape,
    escapechar=@,
    extendedchars=true,
    frame=l,
    framerule=.5pt,
    identifierstyle=\color{black},
    inputencoding=latin1,
    keywordstyle=\color{utred60}\bfseries,
    ndkeywordstyle=\color{utred60}\bfseries,
    numbers=left,
    numbersep=1.25em,
    numberstyle=\scriptsize\ttfamily,
    prebreak = \raisebox{0ex}[0ex][0ex]{\ensuremath{\hookleftarrow}},
    stringstyle=\color{utred60}\ttfamily,
    upquote=true,
    showstringspaces=false,
}

\lstset{literate=%
   *{0}{{{\color{darkgray}0}}}1
    {1}{{{\color{darkgray}1}}}1
    {2}{{{\color{darkgray}2}}}1
    {3}{{{\color{darkgray}3}}}1
    {4}{{{\color{darkgray}4}}}1
    {5}{{{\color{darkgray}5}}}1
    {6}{{{\color{darkgray}6}}}1
    {7}{{{\color{darkgray}7}}}1
    {8}{{{\color{darkgray}8}}}1
    {9}{{{\color{darkgray}9}}}1
}

\lstnewenvironment{latex}
    {\lstset{language=[LaTeX]TeX}}
    {}

%----------------------------------------------------------------------------------------
% MARGINAL CAPTIONS
%----------------------------------------------------------------------------------------

\usepackage{sidenotes}
\usepackage{scrextend} % for ifthispageodd

% objective: instead of having the sidenode number in superscript, we want it like "1:"
\makeatletter
\ExplSyntaxOn
\RenewDocumentCommand \sidenotetext { o o +m }
{
  \IfNoValueOrEmptyTF{#1}
    {
      \@sidenotes@placemarginal{#2}{\thesidenote{}:~#3}
  \refstepcounter{sidenote}
}
    {\@sidenotes@placemarginal{#2}{#1{}:~#3}}
}
\ExplSyntaxOff
\makeatother

% optional objective: automatically justify sidecaptions to match the other marginnotes
% captions of marginfigures etc. shall always be raggedright
% solution from: https://tex.stackexchange.com/questions/358010/subfigures-break-figure-numbering-with-sidecaptions-from-sidenotes-package/358012#358012

\makeatletter
% Instead of "justified" you *can* use "outerraged" in the DeclareCaptionStyle below.
% This may create a inconsistent layout, therefore, we stick to justified by default.
\DeclareCaptionJustification{outerragged}{\ifthispageodd{\RaggedRight}{\RaggedLeft}}

\DeclareCaptionStyle{sidecaption}{format=plain,font={scriptsize,sf},labelfont=bf,margin=0pt,singlelinecheck=true,justification=justified}
\DeclareCaptionStyle{marginfigure}{format=plain,font={scriptsize,sf},labelfont=bf,margin=0pt,singlelinecheck=true}
\DeclareCaptionStyle{margintable}{format=plain,font={scriptsize,sf},labelfont=bf,margin=0pt,singlelinecheck=true}
\DeclareCaptionStyle{widefigure}[justification=centering]{format=plain,font=small,labelfont=bf,justification=RaggedRight,singlelinecheck=true,margin={0px,0px},oneside}
\DeclareCaptionStyle{widetable}[justification=centering]{format=plain,font=small,labelfont=bf,justification=RaggedRight,singlelinecheck=true,margin={0px,0px},oneside}
\makeatother


%----------------------------------------------------------------------------------------
% RESET SIDENOTE COUNTER AT EVERY CHAPTER
%----------------------------------------------------------------------------------------

\let\oldchapter\chapter
\def\chapter{%
  \setcounter{sidenote}{1}%
  \oldchapter
}


%----------------------------------------------------------------------------------------
% SYMBOLS
%----------------------------------------------------------------------------------------

\usepackage{pifont}
\let\oldding\ding% Store old \ding in \oldding
\renewcommand{\ding}[2][1]{\scalebox{#1}{\oldding{#2}}}% Scale \oldding via optional argument
% usage \ding{number} or |ding[factor]{number}


%----------------------------------------------------------------------------------------
% ITEMIZE AND ENUMERATE ENVIRONMENTS
%----------------------------------------------------------------------------------------

\renewcommand{\labelitemi}{\color{utred}{\scalebox{0.8}{\raisebox{0.2ex}{$\blacktriangleright$}}}}
\renewcommand{\labelitemii}{\textbullet}
\usepackage{enumitem}
\setlist[itemize]{parsep=0.8\parskip,left=0pt,topsep=0pt,partopsep=0pt}
\setlist[enumerate]{parsep=0.8\parskip,left=0pt,topsep=0pt,partopsep=0pt}
\setlist[description]{parsep=0.8\parskip,left=0pt,topsep=0pt,partopsep=0pt}


%----------------------------------------------------------------------------------------
% SET PDF METADATA
%----------------------------------------------------------------------------------------

\AtBeginDocument{
\hypersetup{pdftitle=\ttitle} % Set the PDF's title to your title
\hypersetup{pdfauthor=\authorname} % Set the PDF's author to your name
%\hypersetup{pdfkeywords=\keywordnames} % Set the PDF's keywords to your keywords
}
